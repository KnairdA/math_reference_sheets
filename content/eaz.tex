\newcommand{\Primes}{\mathbb{P}}

\section*{Teilbarkeit}

Sei $n \in \N$. $d \in \N$ ist Teiler von $n$, falls $\exists t \in \N : d \cdot t = n$. Man schreibt $d \mid n$. $n$ ist Vielfaches von $d$.

Die Menge aller Teiler ist endlich da $\forall d \mid n : d \leq n$.

Der \emph{größte gemeinsame Teiler} von $m$ und $n$ wird notiert als $ggT(m,n)$ oder $(m,n)$.

Das \emph{kleinste gemeinsame Vielfache} ist $kgV(m,n)$.

Zahlen $m, n \in \N$ heißen \emph{teilerfremd}, wenn $1 \in \N$ der einzige gemeinsame Teiler ist.

\subsection*{$ggT$ als Linearkombination}

Für $m, n \in \Z$ ex. $c, d \in \Z$ s.d. $mc + nd = ggT(m,n)$

\subsection*{Division mit Rest}

$\forall k \in \Z, n \in \N \exists! d \in \Z, r \in \{0,\cdots,n-1\}: k = dn+r$.

$r$ ist Rest der Division von $k$ durch $n$.

\section*{Primzahlen}

Eine \emph{Primzahl} ist ein $1 < n \in \N$ welches keinen natürlichen Teiler außer $1$ und $p$ hat.

$\Primes = \{ n \in \N | n > 1, \forall d,t < n : d \cdot t \neq n \}$

\subsection*{Fundamentalsatz der Arithmetik}

Jedes $n \in \N$ lässt sich eindeutig als sortiertes Produkt von Primzahlen schreiben.

\subsection*{$p$-adische Bewertung}

Sei $p \in \Primes$. Dann:

$\forall 0 \neq k \in \Z \exists! v_p(k) \in \N_0 : p^{v_p(k)} \mid k \land p^{v_p(k)+1} \nmid k$

Insb.: $k = \pm \displaystyle\prod_{p \in \Primes} p^{v_p(k)}$

\vspace*{-4mm}
\begin{align*}
\forall k, l \in \Z: v_p(k+l)   &\geq \min\{v_p(k),v_p(l)\} \\
             v_p(k \cdot l) &= v_p(k) + v_p(l)
\end{align*}

Weiterhin gilt für $a, b \in \N$:

$b \mid a \iff \forall p \in \Primes : v_p(b) \leq v_p(a)$

$ggT(a,b) = \displaystyle\prod_{p \in \Primes} p^{e_p}$ mit $e_p = \min\{v_p(a),v_p(b)\}$

$kgV(a,b) = \displaystyle\prod_{p \in \Primes} p^{f_p}$ mit $f_p = \max\{v_p(a),v_p(b)\}$

\subsection*{Kleiner Satz von Fermat}

Sei $p \in \Primes, c \in \Z$. Dann gilt $p \mid c^p - c$.

\subsection*{Primzahlverteilung}

Sei $k \in \N$. Dann ex. $M \in \N$ s.d. zwischen $M$ und $M+k$ keine Primzahl liegt.

$\forall \epsilon > 0 \exists x_0 \in \R \forall x \geq x_0 \exists p \in \Primes : p \in [x,(1+\epsilon)x]$

\vspace*{2mm}

Die Funktion $\pi(x) := \#\{ p \in \Primes | p \leq x \}$ zählt die Anzahl der Primzahlen unterhalb $x \in \R$.

Der \emph{Primzahlsatz} besagt: $\lim_{x \to \infty} \pi(x) \cdot \frac{\log{x}}{x} = 1$.

Der \emph{Dichtheitssatz} besagt: Die Menge aller Brüche $p/l$ mit $p, l \in \Primes$ liegt dicht in $\R_{\geq 0}$.

\section*{Magmen}

Ein Magma ist Menge mit Verknüpfung $(M, \star)$ wobei $\star : M \times M \to M$ eine Abbildung ist.

\vspace*{2mm}

Ein Magma ist \emph{assoziativ} gdw.:

$\forall l, m, n \in M : ( l \star m ) \star n = l \star ( m \star n )$

Ein Magma ist \emph{kommutativ} gdw.:

$\forall m, n \in M : m \star n = n \star m$

Ein assoziatives Magma heißt \emph{Halbgruppe}.

Ein assoziatives Magma mit beiseitigem Neutralelement heißt \emph{Monoid}.

\subsection*{Magmenhomomorphismen}

Seien $(M,\star)$, $(N,\circ)$ Magmen.

$\Phi : M \rightarrow N$ ist Magmenhomomorphismus, wenn:

$\forall m_1, m_2 \in M : \Phi(m_1 \star m_2) = f(m_1) \circ f(m_2)$

\subsection*{Untermagmen}

$U \subseteq M$ ist Untermagma gdw.: $U \star U \subseteq U$.

\vspace*{2mm}

$\cap_{i \in I} U_i$ ist Untermagma von $M$.

Für $X \subseteq M$ ist $\langle X \rangle_{\text{Magma}}$ Schnitt aller Untermagmen $U$ von $M$ mit $X \subseteq U$.

$\langle X \rangle_{\text{Magma}}$ heißt Magmenerzeugnis von $X$ in $M$.

\subsection*{Untermonoide}

Ein Untermonoid eines Monoids $M$, d.h. eines assoziativen Magmas mit Neutralelement, ist ein Untermagma mit beidseitigem Neutralelement.

\subsection*{Symmetrische Gruppen}

$\text{Sym}(D) := \{ \sigma \in \text{Abb}(D,D) | \sigma \text{ ist bijektiv}\}$

$\text{Sym}(D)$ ist Untermagma von $\text{Abb}(D,D)$.

$S_d$ für $d \in \N$ ist die aus genau $d$ Elementen bestehende symmetrische Gruppe.

\section*{Gruppen}

Eine Gruppe ist ein assoziatives Magma $(M, \star)$ mit beidseitig neutralem Element $e$ und mindestens einer Inversen bzgl. $\star$ für $\forall m \in M$.

Eine Gruppe heißt \emph{kommutativ} bzw. \emph{abelsch} wenn sie als Magma kommutativ ist.

\subsection*{Untergruppen}

Eine Untergruppe ist ein Untermagma $\emptyset \neq U \leq G$ welches unter Inversenbildung abgeschlossen ist.

$U \neq \emptyset \implies x \in U \implies x^{-1} \in U \implies e_G \in U$

$U \leq G \iff \emptyset \neq U \subseteq G \land \forall x, y \in U : xy^{-1} \in U$

Schnitt von Untergruppen ist selbst Untergruppe.

\subsection*{Gruppenerzeugnis}

Der Schnitt aller ein $M \subset G$ beinhalten Untergruppen wird geschrieben als $\langle M \rangle$ und bezeichnet als (Gruppen-)Erzeugnis von $M$.

$\langle M \rangle = \{ x_1 \cdots x_k | k \in \N_0, \forall i \leq k : x_i \in M \lor x_i^{-1} \in M \}$

\subsection*{Zyklische Gruppen}

Gruppe $G$ ist \emph{zyklisch}, wenn $\exists a \in G : G = \langle a \rangle$.

$\forall n \in \N : [1]$ erzeugt $\Z / n\Z$.

$\langle g \rangle = \{ g^k | k \in \Z \}$ ist von $g$ erzeugte zyklische Grp.

\subsubsection*{Ordnung}

Die Ordnung einer Gruppe ist ihre Kardinalität. Die Ordnung eines $g \in G$ ist die Ordnung der von $g$ erzeugten Untergruppe.

Hat $\langle g \rangle$ endliche Ordnung so $\exists k \in \N : g^k = e_G$.

\subsubsection*{Satz von Lagrange}

Sei $G$ endliche Gruppe und $H \leq G$. Dann ist die Ordnung von $H$ ein Teiler der Ordnung von $G$.

\subsubsection*{Index}

Sei $g_1 \sim g_2 := g_1 g_2^{-1} \in H$ Äquivalenzrel. auf $G$.

Die Äquivalenzklasse von $g \in G$ ist definiert als: $[g] = Hg := \{ hg | h \in H\}$.

Für Gruppen $H \leq G$ heißt die Anzahl der Äquivalenzklassen bzgl. $\sim$ Index von $H$ in $G$, geschrieben als $(G : H)$.

Entsprechend gilt für endl. Grp.: $\#G = \#H \cdot (G : H)$.

\subsubsection*{Primzahlordnung einer Gruppe}

In jeder endlichen Gruppe ist Ordnung jedes Elements ein Teiler der Gruppenordnung. Daraus folgt, dass jede Gruppe mit Primzahlordnung eine zyklische Gruppe ist. $G = \langle g \rangle \iff g \neq e_G$

\subsection*{Normalteiler}

Eine $N \leq G$ ist Normalteiler, falls $\forall n \in N, g \in G : gng^{-1} \in N$ gilt. d.h. $N$ ist invariant unter allen inneren Automorphismen.

Es gilt für Normalteiler $N$: $\forall g \in G : gNg^{-1} = N$

Ist $U \leq G$ ein Normalteiler, so schreibt man $U \triangleleft G$.

Untergruppen abelscher Gruppen sind normal.

\subsection*{Einfachheit}

Die nichttriviale Gruppe $G$ heißt \emph{einfach}, wenn sie keine Normalteiler außer $G$ und $\{e_G\}$ besitzt.

\subsection*{Nebenklassen}

Seien $U \leq G$ Gruppen. Dann sind $g, h \in G$ \emph{kongruent modulo $U$}, wenn $g^{-1}h \in U$. Diese Relation bildet Äquivalenzklassen $gU = \{gu | u \in U\}$.

Diese Äquivalenzklassen heißen \emph{Linksnebenklassen} nach $U$, die Menge aller Nebenklassen heißt \emph{Faktorraum} $G/U$.

$\pi_U : G \rightarrow G/U, g \mapsto gU$ ist kanonische Projektion.

\subsection*{Faktorgruppen}

Sei $N \triangleleft G$. $(gN) \cdot (hN) := ghN$ definiert auf $G/N$ eine wohldefinierte Verknüpfung. $G/N$ ist mit dieser Verknüpfung die \emph{Faktorgruppe von $G$ modulo $N$}.

Die kanonische Projektion $\pi_N$ ist Gruppenhomomorphismus mit Kern $N$. Jeder Normalteiler kann also als Kern eines Gruppenhomomorphismus realisiert werden.

\section*{Gruppenhomomorphismen}

Seien $(G,\star)$, $(H,\circ)$ Gruppen.

$f : G \rightarrow H$ ist Gruppenhomomorphismus, wenn:

\begin{enumerate}[label=(\alph*)]
	\item $\forall x, y \in G : f(x \star y) = f(x) \circ f(y)$
	\item $f(e_G) = e_H$
	\item $\forall x \in G : f(x^{-1}) = f(x)^{-1}$
\end{enumerate}

Ist $f : G \rightarrow H$ ein Magmenhomomorphismus gilt:

\begin{enumerate}[label=(\alph*)]
	\item $f$ ist Gruppenhomomorphismus
	\item $f^{-1}(\{e_H\}) \leq G$
	\item $f(G) \leq H$
	\item $f$ ist injektiv $\iff f^{-1}(\{e_H\}) = \{e_G\}$
\end{enumerate}

\subsection*{Kern}

Sei $f : G \rightarrow H$ Gruppenhomomorphismus.

Dann heißt $f^{-1}(\{e_H\}) \leq G$ Kern von $f$.

\subsection*{Konjugation}

Sei $G$ Gruppe, $g \in G$ fest gewählt.

$\kappa_g : G \rightarrow G, x \mapsto gxg^{-1}$ heißt \emph{Konjugation} und ist Gruppenautomorphismus. $x, y \in G$ heißen \emph{zueinander konjugiert}, wenn $\exists g \in G : y = gxg^{-1}$.

\subsection*{Freie Gruppe}

Sei $S$ eine Menge. $F$ ist eine \emph{freie Gruppe über $S$} mit Abbildung $f : S \rightarrow F$ wenn für beliebige Gruppen $G$, Abbildungen $\varphi : S \rightarrow G$ genau ein Gruppenhomomorphismus $\Phi : F \rightarrow G$ existiert, für den $\forall s \in S : \varphi(s) = \Phi(f(s))$ gilt.

\section*{Gruppenoperationen}

Sei $(G,*)$ Gruppe, $M$ Menge.

$\circ : G \times M \rightarrow M$ ist Gruppenoperation, wenn:

\begin{enumerate}[label=(\alph*)]
	\item $\forall m \in M : e_G \circ m = m$
	\item $\forall m \in M, g_1, g_2 \in G : g_1 \circ ( g_2 \circ m ) = ( g_1 * g_2 ) \circ m$
\end{enumerate}

Es gilt: $\forall \Phi \in Hom(G,Sym(M)) : g \circ m := \Phi(g)(m)$ ist Operation von $G$ auf $M$.

Für jede Operation $\circ$ von $G$ auf $M$ ex. ein $\Phi \in Hom(G,M)$ s.d. $\circ$ so konstruiert werden kann.

\subsection*{Bahnen}

Auf $M$ definiert $m_1 \sim m_2 := \exists g \in G : m_1 = g \circ m_2$ eine Äquivalenzrelation.

Ihre Äquivalenzklassen werden \emph{Bahnen} oder \emph{Orbiten} genannt. d.h. $G \circ m = \{ g \circ m | g \in G \}$ ist Bahn.

\subsubsection*{Transitivität}

Operation mit genau einer Bahn heißt \emph{transitiv}. d.h. $\exists m_0 \in M \forall m \in M \exists g \in G : m = g \circ m_0$.

\subsubsection*{Stabilisator}

$Stab_G(m) := \{ g \in G | g \circ m = m \}$ ist Stab. von $m$ in $G$.

\emph{Fixpunkt} von $G$ auf $M$ ist $m \in M$: $Stab_G(m) = G$.

\subsubsection*{Bahnbilanzformel}

Sei $G$ auf endlichem $M$ operierende Gruppe und $R \subseteq M$ ein Vertretersystem der Bahnen. Dann:

$\#M = \sum_{r \in R} (G : Stab_G(r))$

\section*{Sylowsätze}

Eine endliche Gruppe $G$ heißt \emph{$p$-Gruppe} wenn ihre Kardinalität eine Potenz von $p \in \Primes$ ist.

Eine $U \leq G$ heißt \emph{$p$-Sylowgruppe} wenn ihre Kardinalität gleich der maximalen, die Ordnung von $G$ teilenden, $p$-Potenz ist.

Der Satz von Lagrange liefert so die Maximalität einer $p$-Sylowgruppe unter den $p$-Untergruppen.

\subsection*{Erster Sylowsatz}

Sei $G$ endliche Gruppe, $p \in \Primes$.

Dann $\exists U \leq G : U$ ist $p$-Sylowgruppe.

\subsection*{Zweiter Sylowsatz}

Sei $G$ endliche Gruppe, $p \in Primes$, $\#G = p^e \cdot f$:

\begin{enumerate}[label=(\alph*)]
	\item Jede $p$-Untergruppe von $G$ ist in einer $p$-Sylowgruppe von $G$ enthalten.
	\item Je zwei $p$-Sylowgruppen sind konjugiert.
	\item Die Anzahl der $p$-Sylowgruppen teilt $f$.
	\item Die Anzahl der $p$-Sylowgruppen lässt bei Division durch $p$ Rest $1$.
\end{enumerate}

\section*{Ringe}

Ein \emph{Ring} ist Menge $R$ mit Verknüpfungen $+$ und $*$ s.d. $(R,+)$ abelsche Gruppe mit Neutralelement $0$ ist, $*$ assoziativ ist, neutrales Element $1$ besitzt und die Distributivgesetze gelten:

$\forall a, b, c, d \in R : (a+b)*c = ac+bc \land a*(c+d) = ac + ad$

Ist $*$ kommutativ, heißt $R$ kommutativer Ring.

\subsection*{Ringhomomorphismen}

$\Phi : R \to S$ ist \emph{Ringhomomorphismus} zwischen Ringen $R$ und $S$, wenn es bzgl. $+$ und $*$ ein Magmenhomomorphismus ist und $\Phi(1_R) = 1_S$ gilt.

\subsection*{Einheitengruppe}

$R^\times := \{ r \in R : \exists r^{-1} \in R : r r^{-1} = r^{-1} r = 1_R \}$

$(R^\times, *)$ ist Einheitengruppe.

\subsection*{Teilringe}

Ein \emph{Teilring} von Ring $R$ ist $T \subseteq R$ s.d. $T$ bzgl. $+$ Untergruppe und bzgl. $*$ Untermonoid von $R$ ist.

\subsection*{Nullteiler}

$a \in R$ ist \emph{Nullteiler} in Ring $R$, wenn:

$\exists b \in R, b \neq 0 : ab = 0 \lor ba = 0$

Ist $0$ einziger Nullteiler in $R$, so ist $R$ \emph{nullteilerfrei}.

$R$ heißt \emph{Integritätsbereich}, wenn $R$ kommutativ und nullteilerfrei ist.

Teilringe von Integritätsbereichen sind integer.

\subsection*{Charakteristik}

Sei $R$ Ring. Dann $\exists! \Phi \in Hom_{Ring}(\Z,R)$.

Sei $n \in \N_0$ nichtnegativer Erzeuger des Kerns von $\Phi$. Dann heißt $n$ die Charakteristik $char(R)$ von $R$.

Die Charakteristik eines nullteilerfreien Rings $R$ ist entweder $0$ oder Primzahl $p \in \Primes$.

\subsection*{Ideale}

Ein \emph{Ideal} in Ring $R$ ist $I \subseteq R$ s.d. $(I,+) \leq R$ und $\forall x \in I, r \in R : xr \in I \land rx \in I$.

Kerne von Ringhomomorphismen sind ideal und Ideale sind Kerne von Ringhomomorphismen.

\subsection*{Körper}

Ring $R$ heißt Körper, wenn $R$ kommutativ ist und $0 \neq 1$ sowie $R^\times = R \setminus \{0\}$ gelten. Ein Körper ist insb. ein Integritätsbereich.

\subsection*{Chinesischer Restsatz}

Seien $M, N \in \N$ teilerfremd, dann gibt es einen Isomorphismus von Ringen:

$\Z/(MN\Z) \to \Z/M\Z \times \Z/N\Z$

\subsubsection*{Algebraischer Chinesischer Restsatz}

Seien $R$ kommutativer Ring, $I, J$ Ideale in $R$ s.d. $I + J = R$. Dann existiert ein Isomorphismus:

$\Phi : R/(I \cap J) \to R/I \times R/J$

\section*{Moduln}

Sei $R$ Ring. Ein $R$-Modul ist eine abelsche Gruppe $M$ mit Abbildung $\cdot : R \times M \to M$ s.d.:

\vspace*{-4mm}
\begin{alignat*}{3}
	&\forall r, s \in R, m \in M &&: (r+s)\cdot m &&= r\cdot m + s\cdot m \\
	&\forall r \in R, m, n \in M &&: r \cdot (m+n) &&= r\cdot m + r\cdot n \\
	&\forall r, s \in R, m \in M &&: (rs)\cdot m &&= r\cdot(s\cdot m) \\
	&\forall m \in M &&: 1\cdot m &&= m
\end{alignat*}

Diese Bedingungen sind von VRäumen bekannt.

\subsection*{Untermoduln}

Sei $M$ ein $R$-Modul und $U \subseteq M$.

Dann ist $U$ \emph{Untermodul} von $M$, wenn $U$ additive Untergruppe ist und unter der skalaren Multiplikation $\cdot$ mit Elementen aus $R$ invariant ist:

$U \leq M \land \forall r \in R, u \in U : r \cdot u \in U$

\section*{Polynomringe}

Für kommutativen Ring $R$ ist definiert:

$R[X] := \left\{ \displaystyle\sum_{i=0}^d r_i X^i \middle| d \in \N_0, r_i \in R \right\}$

$(R[X], +, *)$ ist kommutativer Ring.

Für $f, g \in R[X]$ gilt:

\begin{enumerate}[label=(\alph*)]
	\item $deg(f+g) \leq \max(deg(f),deg(g))$
	\item $deg(f*g) \leq deg(f) + deg(g)$
	\item Für nullteilerfreie $R$ gilt in (b) Gleichheit
\end{enumerate}

Ist $R$ nullteilerfrei so ist auch $R[X]$ nullteilerfrei und es gilt $(R[X])^\times = R^\times$.

\subsection*{Polynomdivison}

Sei $R$ kommutativer Ring, $f, g \in R[X]$ und $g \neq 0$ mit Einheit als Leitkoeffizient.

Dann $\exists h, r \in R[X] : f = gh+r$ mit $deg(r) < deg(g)$.

\section*{Algebren}

\newcommand{\A}{\mathcal{A}}

Eine $R$-Algebra über Ring $R$ ist Ring $\A$ mit Ringhomomorphismus $\sigma : R \to \A$ s.d. $\forall r \in R, a \in \A : \sigma(r) \cdot a = a \cdot \sigma(r)$ gilt. d.h. $\sigma(r)$ kommutiert mit $a$.

$\sigma$ ist \emph{Strukturhomomorphismus} von $\A$.

$\A$ ist ein $R$-Modul mit Vorschrift $(r,a) \mapsto \sigma(r) \cdot a$.

Die Multiplikation in $\A$ ist bilinear.

Insb. gilt $\forall r, s \in R : \sigma(r)\sigma(s) = \sigma(s)\sigma(r)$

\subsection*{Zentrum}

Für Ring $A$ ist das \emph{Zentrum} definiert als:

$Z(A) := \{ r \in A | \forall a \in A : ra=ar \}$

$Z(A)$ ist Teilring von $A$ und zugleich größter Teilring $R$ s.d. $A$ durch die Inklusion von $R$ nach $A$ zu einer $R$-Algebra wird.

\vspace*{2mm}

Für bel. kommutative Ringe $R$ ist $R[X]$ eine $R$-Algebra vermöge $\sigma : R \to R[X], r \mapsto r = rX^0$.

\subsection*{Algebrenhomomorphismen}

Seien $(A, \sigma), (B, \tau)$ $R$-Algebren.

Ein Ringhomomorphismus $\Phi : A \to B$ ist zugleich Algebrenhomomorphismus, wenn $\Phi \circ \sigma = \tau$ gilt.

\section*{Quotientenkörper}

Sei $R$ Integritätsbereich. Dann ex. Körper $Q$ mit Teilring $R$ und Eigenschaften:

Ist $K$ bel. Körper und $\phi : R \to K$ injektiver Ringhomomorphismus, dann lässt sich $\phi$ zu einem Ringhomomorphismus $\tilde\phi : Q \to K$ fortsetzen.

Der Körper $Q$ heißt \emph{Quotientenkörper} von $R$.

\vspace*{2mm}

Der Quotientenkörper von $\Z$ ist $\Q$.

Der Quotientenkörper von $K[X]$ ist Körper rationaler Funktionen $K(X) := \{ \frac{f}{g} | f, g \in K[X], g \neq 0 \}$.

\section*{Quadratische Reste}

Sei $F$ endlicher Körper mit $q$ Elementen und Charakteristik $p > 2$. Ein $a \in F^\times$ ist \emph{Quadrat} in $F$, wenn $\exists b \in F : b^2 = a$.

Das Bild von $F^\times \to F^\times, b \mapsto b^2$ ist Quadratmenge.

\subsection*{Legendre-Symbole}

\newcommand{\legendre}[2]{\left(\frac{#1}{#2}\right)}

Sei $p \geq 3$ Primzahl. Für $a \in \Z$ ist def.:

\vspace*{-2mm}
$$\legendre{a}{p} = \begin{cases}
	0  & p | a \\
	1  & \exists x \in \Z \setminus p\Z : a \equiv x^2 \ (mod \ p) \\
	-1 & \text{sonst}
\end{cases}$$

$\legendre{a}{p}$ ist das \emph{Legendre-Symbol} von $a$ modulo $p$.

\subsubsection*{Berechnen von Legendre-Symbolen}

Sei $a \in \Z, m, n \in \Z : a=mn, p \in \Primes$:

\vspace*{-4mm}
\begin{align*}
	\legendre{a}{p} &= \legendre{a-p}{p} \\
	\legendre{m \cdot n}{p} &= \legendre{m}{p}\legendre{n}{p}
\end{align*}

\pagebreak

\vspace*{-8mm}
\begin{align*}
	\legendre{2}{p} &= (-1)^{\frac{p^2-1}{8}} \\
	\legendre{-1}{p} &= (-1)^{\frac{p-1}{2}}
\end{align*}

Sei $m, n \in \Primes$ mit $l, p \neq 2$:

\vspace*{-4mm}
\begin{align*}
	\legendre{p}{l}\legendre{l}{p} &= (-1)^{\frac{l-1}{2}\cdot\frac{p-1}{2}} \\
	\legendre{p}{l} &= (-1)^{\frac{l-1}{2}\cdot\frac{p-1}{2}} \legendre{l}{p}
\end{align*}

\section*{Teilbarkeit in Ringen}

Sei $R$ kommutativer Ring. $a \in R$ ist Teiler von $b \in R$, falls $\exists c \in R : b = c \cdot a$. Kurz $a | b$ bzw. $a |_R b$.

In $R$ ist Faktor $c$ i.A. nicht eindeutig.

Ist $R$ nullteilerfrei und $a \neq 0$, so ist $c$ eindeutig.

\subsection*{Assoziiertheit}

$a, b \in R$ sind \emph{assoziiert}, wenn $\exists e \in R^\times : b = a\cdot e$.

$a, b \in \Z$ sind also assoziiert, wenn sie bis auf ihr Vorzeichen übereinstimmen.

Assoziiertheit ist eine Äquivalenzrelation auf $R$.

Die Äquivalenzklasse von $a \in R$ ist $a \cdot R^\times$.

\subsubsection*{Ordnungsrelation}

Sei $R$ kommutativ und nullteilerfrei. Dann ist

$aR^\times \preccurlyeq bR^\times \iff a | b$

Ordnungsrelation auf der Menge der Assoziiertenklassen.

\subsubsection*{$ggT$ in Ringen}

$g \in R$ ist \emph{größter gemeinsamer Teiler} von $a, b \in R$, wenn $g$ gemeinsamer Teiler ist und alle gemeinsamen Teiler von $a, b$ auch $g$ teilen.

$a, b \in R$ sind \emph{teilerfremd}, wenn die Einheiten in $R$ die einzigen gemeinsamen Teiler sind.

\spacing

$ggT(a,e)$ für $a \in R, e \in R^\times$ ist Assoziiertenklasse von $1$, also $R^\times$. $ggT(a,0) = a\cdot R^\times$ da alles $0$ teilt.

Ist $d$ ein gemeinsamer Teiler von $a, b$, dann gilt auch $d |(ax+by)$ für $x,y \in R$.

\section*{Hauptidealringe}

$\{ax+by | x,y \in R\}$ ist Ideal in $R$.

$I \subseteq R$ ist \emph{Hauptideal}, wenn $\exists g \in I : I = Rg$. Die Menge aller Vielfachen von $g$ in $R$ wird geschrieben als $(g) := Rg$.

\spacing

Ein nullteilerfreier kommutativer Ring $R$, in dem jedes Ideal ein Hauptideal ist, heißt \emph{Hauptidealring}.

\subsubsection*{Assoziiertenklassen und Ideale}

Sei $R$ ein solcher Hauptidealring. Dann:

\begin{enumerate}[label=(\alph*)]
	\item $g, h \in R$ sind Erzeuger des selben Hauptideals $Rg = Rh \iff g$ und $h$ assoziiert sind.
	\item $\forall \emptyset \neq S \subseteq R \exists m \in S : m$ ist bzgl. Teilbarkeit minimal.
\end{enumerate}

\subsection*{Chinesischer Restsatz für Hauptidealringe}

Seien $R$ Hauptidealring, $r, s \in R$ teilerfremd (d.h. $1=rx+sy$ für geeignete $x, y \in R$). Dann gilt für Ideale $I = Rr, J = Rs$ der Chinesische Restsatz s.d.:

\vspace*{-2mm}
$$R/(Rrs) \cong R/(Rr) \times R/(Rs)$$

$\forall a, b \in R \exists x \in R : x \equiv a \ (mod \ Rr) \land x \equiv b \ (mod \ Rs)$

\subsection*{Arithmetik in Hauptidealringen}

Sei $R$ kommutativer Ring.

$m \in R$ ist \emph{irreduzibel}, wenn $m \notin R^\times$ und $\forall a, b \in R: m = ab \implies a \in R^\times \lor b \in R^\times$.

$p \in R$ ist \emph{Primelement}, wenn $p \notin R^\times$ und $\forall a, b \in R: p | ab \implies p | a \lor p | b$.

\spacing

Die Irreduzibilität eines $m \in R$ heißt, dass die Assoziiertenklasse $mR^\times$ in $R$ unter Klassen $\neq R^\times$ bzgl. der Teilbarkeitsordnungsrelation minimal ist. Jeder Teiler von $m$ ist entweder Enheit oder zu $m$ assoziiert.

\spacing

Sei $R$ nullteilerfreier kommutativer Ring:

\begin{enumerate}[label=(\alph*)]
	\item Primelement $\neq 0$ in $R$ ist irreduzibel.
	\item $R$ ist Hauptidealring \\ $\implies $ irreduzibles $R$-Element ist auch prim.
\end{enumerate}

\subsubsection*{Primzerlegung in Hauptidealringen}

Sei $R$ Hauptidealring, $\Primes_R$ Vertretersystem der Assoziiertenklassen von Primelementen $\neq 0$. Dann:

\vspace*{1mm}

$\forall r \in R \setminus \{0\} : r$ ist assoziiert zu Produkt endlich vieler Elemente in $\Primes_R$.

Sind $s, t \in \N_0, p_1,\dots,p_s,q_1,\dots,q_t \in \Primes_R$ s.d. Einheiten $\delta, \epsilon \in R^\times$ ex. mit $r = \delta \cdot p_1 \cdot \dots \cdot p_s = \epsilon \cdot q_1 \cdot \dots \cdot q_t$, so gilt $\epsilon = \delta$, $s = t$ und es gilt bis auf Vertauschung der Faktorreihenfolge $\forall 1 \leq i \leq s : p_i = q_i$.

\subsubsection*{Summen zweier Quadrate}

Ein $n \in \N$ ist als Summe zweier Quadrate von Zahlen $\in \Z$ schreibbar $\iff$ Der quadratfreie Anteil von $n$ hat keinen Primteiler, der bei Division durch $4$ Rest $3$ lässt.

\spacing

$n \in \N$ ist Summe zweier Quadrate gdw. sie die komplexe Norm von einem $a + bi \in \Z[i] \setminus \{0\}$ ist.

\subsection*{Restklassenkörper}

Sei $R$ Hauptidealring aber kein Körper. Der Restklassenring $R/Rg$ ist ein Köper gdw. $g$ irreduzibel ist, da $\forall a \notin Rg : a$ modulo $g$ ist invertierbar.

\vspace*{1mm}

Für $p \in \Primes$ ist $\mathbb{F}_p = \Z/p\Z$ Körper mit $p$ Elementen.

\vspace*{1mm}

Ist $p$ ungerade und $a \in \mathbb{F}_p^\times$ kein Quadrat, dann ist $X^2 - a \in \mathbb{F}_p[X]$ irreduzibel.

$\mathbb{F}_p[X]/(X^2-a)$ ist Körper mit $p^2$ Elementen.

\subsection*{Maximale Ideale}

Ein Ideal $I \subset R$ ist \emph{maximales Ideal}, wenn $I \neq R$ und zwischen $I$ und $R$ kein weiteres Ideal liegt.

$I \subset R$ ist maximal gdw. $R/I$ ein Körper ist.

Weiter $\forall a \in R \setminus I : (a + I) = R/I$ und ein zu $a + I$ inverses Element existiert.

\subsection*{Primideale}

Ein Ideal $I \subset R$ ist \emph{Primideal}, wenn:

$\forall x, y \in R : xy \in I \implies x \in I \lor y \in I$.

\vspace*{1mm}

Ein Ideal $I$ ist Primideal gdw. $R/I$ integer ist, da dann jedes maximale Ideal auch ein Primideal ist.

\vspace*{1mm}

In Hauptidealringen ist jedes Primideal ungleich $(0)$ bereits maximal.

\section*{Körpererweiterungen}

Sei $K$ Körper und $L$ Körper, der $K$ umfasst. $K \subseteq L$ ist dann eine \emph{Körpererweiterung}.

\subsection*{Algebraizität und Transzendenz}

Element $\alpha \in L$ heißt \emph{algebraisch} über $K$, wenn ein Polynom $f \in K[X]$ existiert s.d.: $f \not\equiv 0 \land f(\alpha) = 0$.

\vspace*{1mm}

Element $\alpha in L$ heißt \emph{transzendent} über $K$, wenn es nicht algebraisch über $K$ ist.

\vspace*{1mm}

Körper $L$ ist algebraisch über $K$, wenn alle Elemente von $L$ über $K$ algebraisch sind.

\vspace*{1mm}

Sei $\alpha \in L$ algebraisch über $K$. Das Ideal $I(\alpha) := \{f \in K[X] | f(\alpha) = 0\}$ heißt \emph{Verschwindungsideal} und ist nicht das Nullideal im Polynomring. Normierter Erzeuger von $I(\alpha)$ ist das \emph{Minimalpolynom} von $\alpha$.

\subsection*{Adjunktion}

Der kleinste Teilkörper von $L$ welcher $K$ und geg. $\alpha \in L$ enthält, heißt $K(\alpha)$ d.h. \emph{$K$ adjungiert alpha}.

Für jede Teilmenge $A \subseteq L$ existiert ein kleinster Teilkörper welcher $K$ und $A$ enthält.

\subsection*{Algebraische Erweiterung}

Sei $K \subseteq L$ Körpererweiterung. Dann gelten:

\begin{enumerate}[label=(\alph*)]
	\item $\alpha \in L$ ist algebraisch $\iff$ Dimension von $K(\alpha)$ als $K$-Vektorraum ist endlich.
	\item Menge aller über $K$ algebraischen $\alpha \in L$ ist Teilkörper von $L$.
	\item $K \subseteq L$ und $L \subseteq M$ sind algebraische Körpererweiterungen $\implies K \subseteq M$ ist algebraische Körpererweiterung.
\end{enumerate}

\subsection*{Grad der Körpererweiterung}

Sei $K \subseteq L$ Körpererweiterung. Die Dimension von $L$ als $K$-Vektorraum heißt \emph{Grad von $L$ über $K$}.

Geschrieben $[L : K]$.

$\alpha \in L$ ist algebraisch $\iff [K(\alpha) : K] < \infty$.

\vspace*{1mm}

Sind $K \subseteq L \subseteq M$ endliche Körpererweiterungen so gilt: $[M : K] = [M : L] \cdot [L : K]$.
