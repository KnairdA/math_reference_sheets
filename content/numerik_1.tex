\section*{Gleitkomma-Arithmetik}

Für $e_{min}, e_{max} \in \mathbb{Z}$, $e_{min} < e_{max}$ ist ein Gleitkommasystem wie folgt definiert:

\vspace*{-4mm}
\begin{align*}
\mathcal{F} &= \mathcal{F}(\beta,t,e_{min},e_{max}) \\
            &= \{ \pm m \beta^{e-t} | m \in \N, \beta^{t-1} \leq m \leq \beta^t - 1 \lor m = 0, \\ & \hspace*{16mm}e_{min} \leq e \leq e_{max} \}
\end{align*}

$x \in \mathcal{F} \setminus \{0\} \Rightarrow \beta^{e_{min}-1} \leq |x| \leq \beta^{e_{max}}(1-\beta^{-1})$.

\subsection*{Normalisierte Darstellung}

Für $d_1 \neq 0$, $0 < d_1 \leq \beta - 1$:

$x=\pm \beta^e ( \frac{d_1}{\beta^1} + \frac{d_2}{\beta^2} + \cdots + \frac{d_t}{\beta^t} ) =: \pm 0.d_1 d_2 \cdots d_t \cdot \beta^e$

\subsection*{Relative Maschinengenauigkeit}

$fl(x) \in \mathcal{F}$ ist die $x \in \R$ am nächsten liegende Gleitkommazahl.

Für relative Maschinengenauigkeit $\epsilon := \frac{1}{2} \beta^{1-t}$:

$\frac{|fl(x)-x|}{|x|} < \epsilon$, $\frac{|fl(x)-x|}{|fl(x)|} \leq \epsilon$

\subsection*{Arithmetische Grundoperationen}

Für $x, y \in \mathcal{F}$ sind Operationen $o \in \{x,-,*,\div\}$ bzgl. eines Gleitkommasystems definiert:

$\tilde o(x,y) := fl(o(x,y))$

Zu beachten ist hier die Ungültigkeit der Assoziativ- und Distributivgesetze.

\subsection*{Kondition mathematischer Probleme}

Die Konditionszahl eines mathematischen Problems $(f, x)$ ist die kleinste Zahl $\kappa_f(x) \geq 0$ mit:

\vspace*{-4mm}
$$\frac{\|f(x + \Delta x) - f(x) \|_Y}{\|f(x)\|_Y} \leq \kappa_f(x) \frac{\|\Delta x\|_X}{\|x\|_X} + o(\|\Delta x \|_X)$$

Für $\|\Delta x\|_X \rightarrow 0$. Ein Problem $(f, x)$ ist gut konditioniert für \emph{kleine} und schlecht konditioniert für \emph{große} Konditionszahlen $\kappa_f(x)$.

\subsubsection*{Kondition stetig differenzierbarer Fkt.}

Für $f \in C^1(E, \R^m)$ in Umgebung $E \subseteq \R^n$ von $x$:

\vspace*{-2mm}
$$\kappa_f(x) = \frac{\|f'(x)\| \cdot \|x\|_X}{\|f(x)\|_Y}$$

\section*{Vektor- und Matrixnormen}

\subsection*{Induzierte Matrixnorm / Operatornorm}

Für Normen $\| \cdot \|_\circ$, $\| \cdot \|_\star$ auf $\K^n$ bzw. $\K^m$ ist eine Matrixnorm $\| \cdot \| : \K^{m \times n} \rightarrow [0,\infty)$ auf dem Vektorraum der $m \times n$-Matrizen definiert:

\vspace*{-4mm}
$$\|A\| := \max_{v \in \K^n \setminus \{0\}} \frac{\|Av\|_\star}{\|v\|_\circ} = \max_{\{v \in \K^n | \|v\|_\circ = 1 \}} \|Av\|_\star$$

\subsubsection*{Eigenschaften}

Für $A \in \K^{m \times n}$ gilt $\forall v \in \K^n : \|Av\|_\star \leq \|A\| \cdot \|v\|_\circ$

Submultiplikativität: $\|AB\| \leq \|A\| \cdot \|B\|$

\subsubsection*{Matrix-$p$-Normen}

Induzierte Matrixnorm bei Wahl der $p$-Normen über $\K^n$ bzw. $\K^m$:

\vspace*{-4mm}
$$\|A\|_p := \max_{\{v \in \K^n | \|v\|_p = 1 \}} \|Av\|_p \text{ für } 1 \leq p \leq \infty$$

\subsubsection*{Spaltensummennorm}

Für $A = (a_1, \cdots, a_n)$ mit $a_j \in \K^m$:

\vspace*{-4mm}
$$\|A\|_1 = \max_{1 \leq j \leq n} \|a_j\|_1 = \max_{1 \leq j \leq n} \sum_{i=1}^m |a_{i,j}|$$

\subsubsection*{Zeilensummennorm}

\vspace*{-4mm}
$$\|A\|_\infty = \max_{1 \leq i \leq m} \sum_{j=1}^n |a_{i,j}|$$

\subsubsection*{Spektralnorm}

Die Matrix-$2$-Norm wird so genannt, da $\|A\|_2 = \sqrt{\lambda_{max}(A^H A)}$ für $\lambda_{max}(A^H A)$ als Bezeichner des größten Eigenwerts von $A^H A \in \K^{n \times n}$.

$\|A\|_2 = \|A^H\|_2$, $\|A^H A\|_2 = \|A\|_2^2$

$\|Q A\|_2 = \|A\|_2$ für unitäre $Q$.

\subsection*{Induzierte Normen}

Für $A \in \R^{n \times n}$ mit $A > 0$ ist $\skp{z,z}_A := \skp{Az,z}_2$ ein Skalarprodukt auf $\R^n$. Dieses induziert die Energienorm $\|z\|_A = \sqrt{\skp{z,z}_A}$.

\subsection*{Kondition einer Matrix}

Für $A \in \K^{n \times n} \in GL_n{\R}$, $\|\cdot\|$ induzierte Matrixnorm:

\vspace*{-4mm}
\begin{align*}
\kappa(A) &= \|A\| \cdot \|A^{-1}\| \\
1 = \|Id\| = \|AA^{-1}\| &\leq \|A\| \cdot \|A^{-1}\| = \kappa(A)
\end{align*}

\section*{Besondere Matrizen}

\subsection*{Diagonaldominante Matrizen}

$A \in \R^{n \times n}$ ist diagonaldominant, falls:

$\forall i \in \{1,\cdots,n\} : |a_{i,i}| > \sum_{j=1,j\neq i}^n |a_{i,j}|$

Insbesondere sind solche $A$ regulär. Gilt nur $\geq$ so heißt die Matrix schwach diagonaldominant.

\subsection*{Positiv definite Matrizen}

$A \in \R^{n \times n}$ ist positiv definit d.h. $A > 0$ falls $A=A^T$ und $\forall x \in \R^n \setminus \{0\} : x^TAx > 0$.

\subsection*{Hessenberg-Matrizen}

Fast obere / untere Dreiecksmatrix wobei 1. untere / obere Nebendiagonale besetzt sein kann.

\section*{Direkte Verfahren zur LGS Lösung}

\subsection*{Cramersche Regel}

Sei $A = (a_{i,j})_{ij} \in GL_n(\R)$, $b \in \R^n$, $A[j] = (a_1, \cdots, a_{j-1}, b, a_{j+1}, \cdots, a_n) \in \R^{n \times n}$, $a_k$ k-ter Spaltenvektor von $A$. Dann bildet $x_j = \frac{det(A[j])}{det(A)}$ für $j = 1, \cdots, n$ die eindeutige Lösung $x \in \R^n$ s.d. $Ax=b$.

Aufgrund des hohen Aufwands von allg. mehr als $(n+1)!$ arithmetischen Operationen ist die Cramersche Regel nur von theoretischer Bedeutung.

\subsection*{Lösung gestaffelter Systeme}

Obere Dreicksmatrizen können mittels Rückwärtssubstitution, untere Dreiecksmatrizen mittels Vorwärtssubstitution in $\mathcal{O}(n^2)$ gelöst werden.

\subsection*{LR-Zerlegung}

$A = LR$ wobei $L$ untere Dreiecksmatrix mit $1$-Diagonale und $R$ obere Dreicksmatrix.

\subsubsection*{Berechnung LR-Zerlegung}

Die LR-Zerlegung existiert insofern die Diagonaleinträge nicht verschwinden. Insbesondere gilt dies für diagonaldominante Matrizen.

\begin{enumerate}
	\item Spaltenweises nullen der der unteren Einträge mittels \emph{Gauß}, Matrizen $L_1, \cdots, L_{n-1}$
	\item $L = L_1^{-1} \cdots L_{n-1}^{-1}$, $R=L_{n-1} \cdots L_1 A$
\end{enumerate}

\subsubsection*{Lösung $Ax=b$ mittels LR-Zerlegung}

\begin{enumerate}
	\item $A=LR$ berechnen
	\item $Lz=b$ Vorwärtssubstitution
	\item $Rx=z$ Rückwärtssubstitution
\end{enumerate}

\subsubsection*{Spaltenpivotsuche}

Die normale LR-Zerlegung ist nur Vorwärts- und nicht Rückwärtsstabil. Dies kann durch Spaltenpivotsuche verbessert werden. Hier wird in jedem Schritt mittels einer Permutationsmatrix immer mit der größten verbleibenden Zeile eliminiert.

\vspace{1mm}

Für alle regulären Matrizen existiert eine Spaltenpivotsuchen LR-Zerlegung so, dass $PA=LR$.

\subsection*{Cholesky-Zerlegung}

Für $A > 0$ existiert untere Dreiecksmatrix $L$ mit positiver Diagonale, so dass $A = LL^T$

\subsection*{QR-Faktorisierung}

Für alle $A \in \R^{m \times n}$ mit $m \geq n$ und $Rang(A)=n$ existiert $A=QR$ mit unitärem $Q \in \R^{m \times m}$ und injektiver oberer Dreiecksmatrix $R$.

\subsubsection*{Householder-Reflexionen}

$$H(v) := Id_m - 2 \frac{vv^T}{v^Tv} = Id_m - 2 \frac{vv^T}{\|v\|_2^2} \text{ für } \forall v \in \R^m \setminus \{0\}$$

Solche Householder-Reflexionen $H(v)$ sind orthogonal, d.h. $H(v)^T=H(v)$ und $H(v)^2=Id_m$.

Wegen $H(v)v=v-2v=-v$ und $\forall w \in spann\{v\}^\perp : H(w)w=w$ ist $H(v)$ Spiegelung an der Hyperebene $spann\{v\}^\perp$.

Solche Reflexionen können durch wiederholte Anwendung Matrizen in obere Dreiecksgestalt überführen:

\vspace{1mm}

Gesucht ist $v \in \R^m$ für $y \in \R^m$ s.d.:

\vspace{-2mm}
$$H(v)y=y - 2 \frac{\skp{v,y}}{\|v\|_2^2}v \overset{!}{=} \alpha e_1$$

Vermeidung von Auslöschung: $\alpha := -sign(y_1)\|y\|_2$

Es ergibt sich mit $v:=y-\alpha e_1$: $H(y-\alpha e_1)y=\alpha e_1$.

\vspace{1mm}

Seien $Q_k$ die sukzessiven, auf $m \times m$ erweiterten, Householder-Reflexionen. Dann gilt:

\vspace{1mm}

$R:=Q_p \cdots Q_1 A$, $Q:=Q_1^T \cdots Q_p^T$ s.d. $A=QR$.

\subsubsection*{Givens-Rotationen}

Mit $c^2 + s^2 = 1, c, s \in \R$ und $l < k$:

$$G(l,k) := \left(\begin{smallmatrix}
1 &           &   &    &   &           &   &   &   &           &   \\
  & \diagdown &   &    &   &           &   &   &   &           &   \\
  &           & 1 &    &   &           &   &   &   &           &   \\
  &           &   &  c &   &           &   & s &   &           &   \\
  &           &   &    & 1 &           &   &   &   &           &   \\
  &           &   &    &   & \diagdown &   &   &   &           &   \\
  &           &   &    &   &           & 1 &   &   &           &   \\
  &           &   & -s &   &           &   & c &   &           &   \\
  &           &   &    &   &           &   &   & 1 &           &   \\
  &           &   &    &   &           &   &   &   & \diagdown &   \\
  &           &   &    &   &           &   &   &   &           & 1
\end{smallmatrix}\right)$$

Wobei $c$ das Diagonalelement der $l$-ten und $k$-ten Zeile, $s$ $k$-tes Element der $l$-ten Zeile, $-s$ $l$-tes Element der $k$-ten Zeile.

Givens-Rotationen sind orthogonal, $G(l,k)A$ unterscheidet sich von $A$ nur in der $l$-ten und $k$-ten Zeile.

\vspace{-4mm}
$$(G(l,k)x)_i = \begin{cases}
	 cx_l + sx_k & i=l \\
	-sx_l + cx_k & i=k \\
	x_i          & \text{sonst}
\end{cases}$$

$\exists \varphi \in (0,2\pi] : c=\cos{\varphi}, s=\sin{\varphi}$ d.h. $G(l,k)$ ist Rotation um $\varphi$ in Ebene $spann\{e_l,e_k\}$.

Ziel: $k$-te Komponente von $x$ nullen für $x_l^2+x_k^2 \neq 0$.

\vspace{-4mm}
\begin{align*}
	|x_l| > |x_k| : &\tau := \frac{x_k}{x_l}, c := \frac{1}{\sqrt{1+\tau^2}}, s := c\tau \\
	|x_l| \leq |x_k| : &\tau := \frac{x_l}{x_k}, s := \frac{1}{\sqrt{1+\tau^2}}, c := s\tau
\end{align*}

Mit einer solchen Givens-Rotation können einzelne Matrixelemente genullt und $A \in \R^{m \times n}$ so sukzessive in eine obere Dreiecksmatrix transformiert werden.

QR mit Householder ist ungefähr doppelt so schnell wie mit Givens. Diese sind daher nur bei strukturierten Matrizen wie Tridiagonal- oder Hessenberg-Matrizen sinnvoll einzusetzen.

\section*{Lineare Ausgleichsprobleme}

Finde $u^* \in \R^n$: $\|Au^*-b\|_2 = \min_{u\in \R^n} \|Au-b\|_2$

Es sind äquivalent:

\begin{enumerate}[label=(\alph*)]
	\item $u^*$ löst Ausgleichsproblem
	\item $u^*$ löst Normalengleichung $A^TAu^*=A^Tb$
	\item $u^*$ erfüllt $Au^* = P_Ab$ mit Ortogonalprojekt. $P_A : \R^m \rightarrow \R^m$ auf Bild von $A$
\end{enumerate}

Das Residuum $Au^*-b$ steht normal zu Bild von $A$.

Ein Ausgleichsproblem ist eindeutig lösbar gdw. $Kern(A) = \{0\}$.

Die Lösung mit minimaler euklidischer Norm wird Minimum-Norm-Lösung $u^+$ genannt.

\subsection*{Singulärwertzerlegung}

Sei $A \in \R^{m \times n}$ mit $r=Rang(A) \leq \min\{m,n\}$.

Dann existiert $A=USV^T$ mit orthogonalen $U \in \R^{m \times m}$, $V \in \R^{n \times n}$ sowie Diagonalmatrix $S \in \R^{m \times n}$.

\section*{Iterative Verfahren zur LGS Lösung}

Ein iteratives Verfahren zur Lösung von $Au=b$ liefert zu Startvektor $u^0 \in \R^n$ eine Folge $\{u^k\}_{k\in \N_0} \subset \R^n$ mittels $\Psi_k : (\R^n)^{k+1} \rightarrow \R^n$ durch $u^{k+1} = \Psi_k(u^0, \cdots, u^k)$.

Ein Verfahren konvergiert falls $\forall u^0 \in \R^n$, $b \in \R^n : \lim_{k\to \infty} u^k = A^{-1}b$

\subsection*{Abbruchkriterium}

Direkte Löser liefern bei exakter Arithmetik nach endlich vielen Schritten $A^{-1}b$. Bei iterativen Lösern gilt im allg. $\forall k \in \N : u^k \neq A^{-1}b$.

Ein Abbruchkriterium mit Toleranz $\tau > 0$ ist z.B. $\|u^m-u^{m-1}\| \leq \tau$ oder das relative Residuum von $u^m$: $\frac{\|Au^m-b\|}{\|b\|} \leq \tau$

\subsection*{Lineare Iterationsverfahren}

$u^{k+1} = u^k + N(b-Au^k) = (Id - NA)u^k + Nb$ zu $u^0 \in \R^n$ wobei $N \in GL_n(\R)$ das jeweilige Verfahren charakterisiert.

$M := Id - NA$ heißt Iterationsmarix.

\subsubsection*{Spektralradius}

$\varrho(M) = \max\{|\lambda| : \lambda \in Spec(M)\}$

\subsubsection*{Konvergenz}

Ein lineares Iterationsverfahren konvergiert gdw. Spektralradius von $M$ $\varrho(M) < 1$.

Ist $\|M\| < 1$ für induzierte Matrixnorm $\|\cdot\|$ dann konvergiert das entsprechende lineare Iterationsverfahren ebenso.

Sind $A$ oder $A^T$ diagonaldominant so konvergieren sowohl das Jacobi- als auch das Gauß-Seidel-Verfahren.

\subsubsection*{Gesamtschritt- / Jacobi-Verfahren}

$$u_i^{k+1} = u_i^k + \frac{1}{a_{i,i}}\left(b_i - \sum_{j=1}^n a_{i,j} u_j^k \right) \text{ für } i = 1, \cdots, n$$

Für $A = D - L - R$:

\vspace{-4mm}
\begin{align*}
	u^{k+1} &= u^k + D^{-1}(b-Au^k) \\
	        &= (Id - D^{-1}A)u^k + D^{-1}b
\end{align*}

Also: $M^{Jac} = Id - D^{-1}A = D^{-1}(L+R)$, $N^{Jac} = D^{-1}$

\subsubsection*{Einzelschritt- / Gauß-Seidel-Verfahren}

$$u_i^{k+1} = u_i^k + \frac{1}{a_{i,i}}\left(b_i - \sum_{j=1}^{i-1} a_{i,j} u_j^{k+1} - \sum_{j=i}^n a_{i,j} u_j^k \right)$$

Für $A = D - L - R$:

$M^{GS} = (D - L)^{-1}R$, $N^{GS} = (D - L)^{-1}$.

\subsection*{Krylov-Raum-Verfahren}

Der Krylov-Raum $m$-ter Ordnung bzgl. $B$ und $v$:

\vspace{-4mm}
$$\mathcal{U}_m(B,v) := spann\{v,Bv,B^2v, \cdots, B^{m-1}v\} \subset \R^n$$

Das Residuuum der $k$-ten Iterierten $u^k$:

\vspace{-2mm}
$$r^k=(I-AN)r^{k-1} = (I-AN)^kr^0$$

Es gilt: $u^k \in u^0 + V_k$ mit $V_k = \mathcal{U}_k(NA,Nr^0)$ sowie $r^0 = b - Au^0$.

Minimaleigenschaft der $k$-ten Iterierten bzgl. Norm $\|\cdot\|_\star$ auf $\R^n$:

\vspace{-2mm}
$$u^k = argmin\{\|v - A^{-1}b\|_\star : v \in V_k\}$$

Ein Krylov-Raum-Verfahren bzgl. einer Norm $\|\cdot\|_\star$ ist nur dann sinnvoll, wenn $u^k$ mit geringem, d.h. im Bereich von zwei Matrix-Vektormultiplikationen liegendem, numerischen Aufwand aus $u^{k-1}$ hervorgeht.

\subsection*{Vorkonditionierer}

Anforderungen: $Nv$ sollte schnell zu berechnen sein, weiterhin sollte $N \approx A^{-1}$ möglichst gelten.

\subsection*{Bezüglich $A > 0$ konjugierte Vektoren}

Vektoren $p, q \in \R^n$ sind konjugiert bzgl. $A > 0$ d.h. $A$-orthogonal, falls $Ap \perp q$, also $\skp{Ap,q}_2=\skp{p,q}_A=0$.

\subsection*{CG-Verfahren}

Das Verfahren der konjugierten Gradienten ist durch die Energienorm charakterisiert und definiert als:

\vspace{-2mm}
$$u^k = argmin\{\|v-A^{-1}b\|_A : v \in V_k\}$$

Für positiv definite $A, N \in \R^{n \times n}$ sowie $b \in \R^n$ liefert das CG-Verfahren nach spätestens $n$ Schritten $A^{-1}b$. Eigentlich ist es bei exakter Arithmetik also ein direkter Verfahren, wird aber durch früheren Abbruch als iteratives Verfahren verwendet.

\subsection*{GMRES-Verfahren}

Das Verfahren des verallgemeinerten minimalen Residuum liefert die Lösung $Ax=b$ für $A \in GL_n(\R)$ und ist charakterisiert durch:

\vspace{-4mm}
\begin{align*}
	u^k :&= argmin\{\|N(b-Av)\|_2 : v \in V_k\} \\
	     &= argmin\{\|NA(A^{-1}b-v)\|_2 : v \in V_k\}
\end{align*}

Das GMRES-Verfahren ist also durch die nicht Skalarprodukt-induzierte  Norm $\|NA\cdot\|_2$ induziert.

\section*{Interpolation}

Interpolation von $f$ mit $\varphi$ s.d. $\varphi(t_i) = f(t_i)$ für $i = 0,\cdots, n$. Approximation von $f$ mit $\varphi$ s.d. $\|\varphi - f\|$ möglichst klein in geeigneter Norm.

\subsection*{Klassische Polynom-Interpolation}

Zu gegebenen Knoten $t_0 < \cdots < t_n$ und Stützwerten $f_i = f(t_i)$ für $i = 0,\cdots,n$ wird Polynom $ \in \Pi_n$ gesucht s.d. $P(t_i)=f_i$ für $i = 0,\cdots,n$.

Zu $n+1$ Stützwerten $f_i$ und paarweise verschiedenen Knoten $t_i$ existiert dabei genau ein solches Interpolationspolynom $P=P(f|t_0,\cdots,t_n) \in \Pi_n$.

\subsection*{Vandermonde-Matrix}

$$\begin{pmatrix}
1      & t_0    & t_0^2  & \cdots & t_0^n  \\
1      & t_1    & t_1^2  & \cdots & t_1^n  \\
\vdots & \vdots & \vdots &        & \vdots \\
1      & t_n    & t_n^2  & \cdots & t_n^n
\end{pmatrix}
\begin{pmatrix}a_0 \\ \vdots \\ \vdots \\ a_n\end{pmatrix} =
\begin{pmatrix}f_0 \\ \vdots \\ \vdots \\ f_n\end{pmatrix}$$

Die Lösung der Vandermonde-Matrix beschreibt $P(f|t_0,\cdots,t_n) \in \Pi_n$, was jedoch sehr aufwändig ist.

\subsection*{Lagrange-Basis}

Basis $\{L_{n,0},\cdots,L_{n,n}\}$ von $\Pi_n$ abhg. $t_0 < \cdots < t_n$ wg. $L_{n,k}(t_i) = \delta_{k,i} = \begin{cases}1 & k=i \\ 0 & sonst\end{cases}$

$$L_{n,k}(t) := \prod_{j=0,j\neq k}^n \frac{t-t_j}{t_k-t_j}$$

Es gilt also: $P(f|t_0,\cdots,t_n)=\sum_{k=0}^n f_k \cdot L_{n,k}$

\vspace{2mm}

Ein Lagrange Polynom zu Stützstelle $t_k$ nimmt an dieser $1$, an allen anderen Stützstellen $0$ an.

\subsubsection*{Lemma von Aitken}

$P = P(f|t_0,\cdots,t_n)(t) =$
\vspace{-2mm}
$$\frac{(t_0-t)P(f|t_i,\cdots,t_n)(t)-(t_n-t)P(f|t_0,\cdots,t_{n-1})(t)}{t_0-t_n}$$

\subsubsection*{Schema von Neville}

Sei $t \in \R$ fest, $P_{i,k}(t)=P_{i,k}=P(f|t_{i-k},\cdots,t_i)(t)$ für $i\geq k$. Dann ist insb. $P_{i,0}=f_i$ und $P_{n,n}=P(f|t_0,\cdots,t_n)(t)$ kann rekursiv mit dem \emph{Schema von Neville} berechnet werden:

\vspace{-4mm}
\begin{align*}
P_{i,k} &= \frac{(t_{i-k}-t)P_{i,k-1} - (t_i-t)P_{i-1,k-1}}{t_{i-k}-t_i} \\
        &= \frac{t-t_{i-k}}{t_i-t_{i-k}} P_{i,k-1} - \frac{t-t_i}{t_i-t_{i-k}} P_{i-1,k-1}
\end{align*}

\subsection*{Tschebyscheff-Knoten}

Für $i = 0,\cdots,n$:

\vspace{-2mm}
$$t_i^{[a,b]} = \frac{b+a}{2} + \frac{b-a}{2} \cos\left(\frac{2i+1}{2n+2} \pi\right)$$

Diese Knotenfolge liegt dichter zu den Intervallgrenzen hin und ergibt eine bessere Interpolation als äquidistante Knoten.

\subsection*{Satz von Faber}

Zu jeder Folge von Knoten $\{t_0^{(n)},\cdots,t_n^{(n)}\}_{n \in \N}$ in $[a,b]$ gibt es ein $f \in C([a,b])$ so, dass $\{P(f|t_0^{(n)},\cdots,t_n^{(n)})\}_{n \in \N}$ für $n \to \infty$ nicht glm. gegen $f$ konvergiert.

\section*{Splines}
