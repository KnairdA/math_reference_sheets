\newcommand{\C}{\mathbb{C}}

\section*{Komplexe Zahlen}

$\C = \{ z = x+iy | x,y \in \R \}$

$\C$ wird via $z = x + iy \mapsto (x,y)$ mit $\R^2$ identifiziert.

\vspace*{-4mm}
$$z \cdot w = \begin{pmatrix} x & -y \\ y & x\end{pmatrix} \begin{pmatrix} u \\ v\end{pmatrix} = \begin{pmatrix} r & 0 \\ 0 & r\end{pmatrix} \begin{pmatrix} \frac{x}{r} & -\frac{y}{r} \\ \frac{y}{r} & \frac{x}{r}\end{pmatrix} \begin{pmatrix} u \\ v\end{pmatrix}$$

wobei $r := \sqrt{x^2 + y^2}$. Es gilt für die orthogonale Matrix $D = \begin{pmatrix} \frac{x}{r} & -\frac{y}{r} \\ \frac{y}{r} & \frac{x}{r}\end{pmatrix}$: $\det D = 1$ d.h. die komplexe Multiplikation ist eine Drehstreckung.

Die Normen von $(\C,|\cdot|)$ und $(\R^2,|\cdot|_2)$ stimmen überein, ebenso Konvergenz-, Stetigkeits- und Offenheitseigenschaften:

$\lim_{n \to \infty} z_n = z$ in $\C \iff \lim_{n \to \infty} Re \ z_n = Re \ z \land \lim_{n \to \infty} Im \ z_n = Im \ z$.

\subsection*{Polardarstellung}

Für $z = x +iy \in \C \setminus \{0\}$ gilt $z = re^{i\phi}$ mit $r = |z|$ und:

\vspace*{-2mm}
$$\phi = \arg z := \begin{cases}
	\arccos \frac{x}{r} & y > 0 \\
	0 & x \in (0,+\infty) \\
	-\arccos \frac{x}{r} & y < 0 \\
	\pi & z \in (-\infty,0)
\end{cases}$$

mit $\phi \in (-\pi, \pi]$. Es gilt für $z = re^{i\phi}, w = se^{i\psi}$:

$z \cdot w = rse^{i(\phi+\psi)} = |z||w|e^{i(\phi+\psi)}$.

\section*{Holomorphie}

Eine Funktion $f : D \to \C$ ist \emph{komplex differenzierbar} in $z_0 \in D$, wenn:

$f'(z_0) := \displaystyle\lim_{z \to z_0, z \in D\setminus\{z_0\}} \frac{f(z)-f(z_0)}{z-z_0} \in \C$ existiert.

Ist $f$ in $\forall z_0 \in D$ komplex differenzierbar, so heißt $f$ \emph{holomorph} auf $D$ mit Ableitung $f' : D \to \C$.

Geschrieben $f \in H(D)$.

\subsection*{Komplexe Ableitung}

$\C \to \C, z \mapsto 1$ und $\C \to \C, z \mapsto z$ sind holomorph.

Seien $f, g : D \to \C$ komplex differenzierbar in $z_0 \in \C$, $f(z_0) \in D' \subseteq \C$ offen, $h : D' \to \C$ in $f(z_0)$ komplex differenzierbar, $\alpha, \beta \in \C$:

\vspace*{-4mm}
\begin{align*}
	(\alpha f + \beta g)'(z_0) &= \alpha f'(z_0) + \beta g'(z_0) \\
	(fg)'(z_0) &= f'(z_0)g(z_0) + f(z_0)g'(z_0) \\
	\left(\frac{1}{f}\right)'(z_0) &= -\frac{f'(z_0)}{f(z_0)^2} \\
	(h \circ f)'(z_0) &= h'(f(z_0))f'(z_0)
\end{align*}

Polynome $p$ und nichtsinguläre rationale Funktionen aus Polynomen sind auf $\C$ holomorph.

\subsubsection*{Konvergenzradius}

Seien $a_k \in \C, k \in \N_0$:

\vspace*{-2mm}
$$\rho = \frac{1}{\overline\lim_{k\to\infty} \sqrt[k]{|a_k|}} \in [0,+\infty]$$

ist der \emph{Konvergenzradius}.

Sei $\rho > 0, c \in \C$. Dann ex. die Potenzreihe:

$f : B(c,\rho) \to \C, z \mapsto \sum_{k=0}^\infty a_k(z-c)^k$.

Diese ist auf $B(c,\rho)$ beliebig oft komplex differenzierbar. Für $n \in \N_0$ hat $f^{(n)}$ den Konvergenzradius $\rho > 0$ und es gilt für $z \in B(c,\rho)$:

\vspace*{-4mm}
$$f^{(n)}(z) = \sum_{k=n}^\infty k(k-1)\cdots(k-n+1)a_k(z-c)^{k-n}$$

Auf diese Weise ergeben sich für $z \in \C$:

\vspace*{-4mm}
\begin{align*}
	\exp(z) &= e^z = \sum_{n=0}^\infty \frac{z^n}{n!} \\
	\sin(z) &= \sum_{n=0}^\infty \frac{(-1)^n}{(2n+1)!} z^{2n+1} \\
	\cos(z) &= \sum_{n=0}^\infty \frac{(-1)^n}{(2n)!} z^{2n}
\end{align*}

Übliche Ableitungs- und Rechenregeln gelten.

\subsection*{Charakterisierung}

Sei $f : D \to \C, D \subseteq \R^2, u = Re \ f, v = Im \ f : D \to \R$.

$f : D \to \R^2, (x,y) \mapsto u(x,y) + iv(x,y) = \begin{pmatrix} u(x,y) \\ v(x,y)\end{pmatrix}$

Es sind dann äquivalent:

\begin{enumerate}[label=(\alph*)]
	\item $f$ ist in $z$ komplex differenzierbar
	\item $f$ ist in $z$ reell differenzierbar und es gelten die \emph{Cauchy-Riemannschen DGL}: \\
		$$\frac{\partial u}{\partial x}(x,y) = \frac{\partial v}{\partial y}(x,y), \frac{\partial u}{\partial y}(x,y) = -\frac{\partial v}{\partial x}(x,y)$$
\end{enumerate}

$f$ hat in $(x,y) \in D \subseteq \R^2$ die \emph{Jacobimatrix}:

$$f'(z) = \begin{pmatrix}
	\frac{\partial u}{\partial x}(x,y) & \frac{\partial u}{\partial y}(x,y) \\
	-\frac{\partial u}{\partial y}(x,y) & \frac{\partial u}{\partial x}(x,y)
\end{pmatrix}$$

Entsprechend ist $f(z)=\overline z$ nirgends komplex differenzierbar, $f(z)=|z|^2$ nur in $0$ komplex differenzierbar und $f(z) = \frac{1}{z}$ holomorph in $\C \setminus \{0\}$.

\subsection*{Biholomorphie}

Sind $U, V \subseteq \C$ offen und nichtleer, $f : U \to V$ bij., $f$ und $f^{-1}$ holomorph. Dann heißt $f$ \emph{biholomorph}, $U$ und $V$ \emph{konform äquivalent}.

\vfill\null
\columnbreak

Sei $f : U \to V$ biholomorph, $z \in U$.

Dann ist $f'(z) \neq 0$ und für $w = f(z)$ gilt:

\vspace*{-2mm}
$$(f^{-1})'(w) = \frac{1}{f'(f^{-1}(w))} = \frac{1}{f'(z)}$$

Weiterhin existieren offene nichtleere $U \subseteq D$ mit $u_0 \in U, V \subseteq \C$ s.d. $\restrictedto{f}{U}$ biholomorph ist, wenn $f \in H(d) \cap C^1(D,\R^2)$, $z_0 \in D$ mit $f'(z_0) \neq 0$ gilt.

\section*{Möbiustransformationen}

Sei $A = \begin{pmatrix} a & b \\ c & d\end{pmatrix} \in \C^{2 \times 2}$ mit $\det A = ad - bc \neq 0$.

Setze $m_A : D_A \to \C, z \mapsto \frac{az+b}{cz+d}$

Mit $D_A = \begin{cases} \C \setminus \{-\frac{d}{c}\} & c \neq 0 \\ \C & c = 0\end{cases}$

\subsection*{Eigenschaften}

\begin{enumerate}[label=(\alph*)]
	\item $m_A$ ist holomorph
	\item $\forall \alpha \in \C \setminus \{0\} : m_{\alpha A} = m_A$
	\item $B \in \C^{2 \times 2}$ mit $\det B \neq 0 \implies m_A \circ m_{B} = m_{AB}$
	\item $m_A(D_A) = D_{A^{-1}}, m_A^{-1} = m_{A^{-1}}$
	\item $m_A : D_A \to D_{A^{-1}}$ ist biholomorph
\end{enumerate}

Alle Möbiustransformationen sind Produkt $m_A = S_1 J S_2$ von affinen Abbildungen $S_j$ und der Inversion $Jz = \frac{1}{z}$. Affine Abbildungen sind Komposition von Translation $Tz=z+\frac{b}{d}$ und Drehstreckung $Dz = \frac{a}{c} z$. $S_j, J, T$ und $D$ sind selbst Möbiustransformationen.

\section*{Potenzen und Wurzeln}

Für $\theta \in (0, \pi]$ ist $\Sigma_\theta := \{ z \in \C \setminus \{0\} | |\arg z| < \theta \}$ der \emph{offene Sektor}.

d.h. $\Sigma_{\pi / 2} = \C_+$ ist die offene rechte Halbebene und $\Sigma_\pi = \C \setminus (-\infty,0]$ die geschlitzte Ebene.

\vspace*{2mm}

$g_a = \{ a + iy | y \in \R \}$ für $a \in \R$ ist \emph{horizontale Gerade}.

$h_b = \{ x + ib | x \in \R \}$ für $b \in \R$ ist \emph{vertikale Gerade}.

\vspace*{2mm}

Die Potenz ist def.: $P_n : \C \to \C, z \mapsto z^n = |z|^n e^{in\phi}$

Sie bildet den Halbstrahl $s_\theta := \{ re^{i\theta} | r > 0 \}$ bijektiv auf den Halbstrahl $s_{n\theta}$ ab.

\subsection*{Hauptzweig der $n$-ten Wurzel}

Sei $n \in \N$ mit $n \geq 2$. Der \emph{Hauptzweig der $n$-ten Wurzel} ist $r_n = p_n^{-1} : \Sigma_\pi \to \Sigma_{\pi/n}$.

Somit ist $\forall w \in \Sigma_\pi$ die $n$-te Wurzel $r_n(w) = z$ das einzige $z \in \Sigma_{\pi/n}$ mit $z^n = w$. Es gelten $r_n(z^n) = z$ und $r_n(w)^n = w$.

\section*{Exponentiale und Logarithmen}

Sei $z = x + iy, x, y \in \R, k \in \Z$. Dann:

\vspace*{-4mm}
\begin{align*}
	\exp(z) &= e^x e^{iy} = e^x(\cos y + i \sin y) \\
	\exp(z) &= \exp(z+2\pi ik) \\
	\exp(z) &= 1 \iff z=2\pi i k
\end{align*}

Für $a, b \in \R$ gilt: $\exp : h_b \to s_b$ ist bijektiv und $\exp : g_a \to \partial B(0,e^a)$ ist surjektiv, nicht injektiv.

\subsection*{Hauptzweig des Logarithmus}

\vspace*{-4mm}
\begin{align*}
	S_r(a_1,a_2) &:= \{ z \in \C | \text{Re } z \in (a_1,a_2) \} \\
	S_i(b_1,b_2) &:= \{ z \in \C | \text{Im } z \in (b_1,b_2) \}
\end{align*}
\vspace*{-6mm}

Sind die \emph{vertikalen und horizontalen Streifen} in $\C$.

\spacing

Der \emph{Hauptzweig des Logarithmus} ist die Abbildung $\log = (\restrictedto{\exp}{s_i})^{-1} : \Sigma_\pi \to S_i$.

$\forall w \in \Sigma_\pi : z = \log(w)$ ist eind. $z \in S_i$ mit $\exp(z) = w$.

Weiter gilt: $\exp : S_i \to \Sigma_\pi$ und $\log : \Sigma_\pi \to S_i$ sind biholomorph mit $\log\exp(z) = z$ für $z \in S_i$ und $\exp\log(w) = w$, $\log'(w) = \frac{1}{w}$ für $w \in \Sigma_\pi$.

\subsection*{Allgemeine Potenz}

Sei $z = re^{i\phi} \in \Sigma_\pi$ mit $r > 0$ und $\phi \in (-\pi,\pi), w = x + iy \in \C$ für $x, y \in \R$. \emph{Allgemeine Potenz} ist def.:

$z^w = \exp(w \log z) = r^x e^{-y\phi} e^{i(x\phi + y \ln r)}$

z.B. $e^w = \exp(w)$ und $i^i = e^{-\pi/2}$.

\spacing

Es gilt $z^{v+w} = z^v z^w$. Ableitungen $\frac{\partial}{\partial z} z^w = wz^{w-1}$ und $\frac{\partial}{\partial w} z^w = \log(w)z^w$ existieren.

\section*{Komplexe Kurvenintegrale}

Fkt $f : [a,b] \to \C$ ist \emph{stückweise stetig}, wenn $\forall t \in [a,b]$ beideitige Grenzwerte in $\C$ ex. und max. endlich viele Unstetigkeitspunkte $t_k \in [a,b]$ ex.

Geschrieben $f \in PC([a,b],\C)$.

Solche Funktionen sind integrierbar:

\vspace*{-4mm}
$$\int_a^b f(t) dt := \int_a^b \text{Re } f(t) dt + i \int_a^b \text{Im } f(t) dt \in \C$$

\subsection*{Hauptsatz}

$f \in PC([a,b],\C)$ ist in $t_0 \in [a,b]$ differenzierbar, wenn $f'(t_0) := \lim_{t \to t_0} \frac{f(t)-f(t_0)}{t-t_0} \in \C$ existiert.

$\iff \text{Re } f, \text{Im } f$ besitzen Ableitungen in $\R$.

Ist $f$ auf $[a,b]$ diffbar und $g, f' \in C([a,b],\C)$. Dann gilt der Hauptsatz:

\vspace*{-3mm}
$$\int_a^b f'(t) dt = f(b) - f(a)$$

$$\exists \frac{d}{dt} \int_a^t g(s) ds = g(t) \text{ für } t \in [a,b]$$

\subsection*{Kurven und Parametrisierungen}

$\gamma \in C([a,b],\C)$ ist \emph{Kurve} oder \emph{Weg} von $\gamma(a)$ nach $\gamma(b)$. $\gamma$ ist \emph{geschlossen}, wenn $\gamma(a)=\gamma(b)$ gilt und einfach, wenn $\gamma$ auf $[a,b)$ injektiv ist.

$\Gamma = \gamma([a,b])$ ist \emph{Bild} oder \emph{Spur} von $\gamma$.

Gilt $\Gamma \subseteq M \subseteq \C$, so ist $\gamma$ Weg in $M$.

$\gamma$ ist auch \emph{Parametrisierung} ihres Bildes $\Gamma$.

\subsection*{Kurvenintegral}

Sei $\gamma \in PC^1([a,b],\C)$ mit Bild $\Gamma = \gamma([a,b])$ und $f \in C(\Gamma,\C)$. Dann ist das \emph{komplexe Kurvenintegral}:

\vspace*{-2mm}
$$\int_\gamma f dz = \int_\gamma f(z) dz := \int_a^b f(\gamma(t))\gamma'(t) dt$$

Die Länge von $\gamma$ ist $l(\gamma) = \int_a^b |\gamma'(t)| dt$.

\subsubsection*{Eigenschaften}

Seien $\gamma, \gamma_1, \gamma_2 \in PC^1([a,b],\C)$ mit Bildern $\Gamma, \Gamma_1, \Gamma_2$ und $f, g \in C(\Gamma,\C), h \in C(\Gamma_1 \cup \Gamma_2, \C), \alpha, \beta \in \C$:

\begin{enumerate}[label=(\alph*)]
	\item $\int_\gamma (\alpha f + \beta g) dz = \alpha \int_\gamma f dz + \beta \int_\gamma g dz$
	\item $|\int_\gamma f dz| \leq \|f\|_\infty l(\gamma)$
	\item $\int_{\gamma_1 \cup \gamma_2} h dz = \int_{\gamma_1} h dz + \int_{\gamma_2} h dz$
	\item $\int_{\gamma^-} f dz = - \int_{\gamma} f dz$
\end{enumerate}

Sei $\gamma \in PC^1([a,b],\C)$ mit Bild $\Gamma$, $f_n, f \in C(\Gamma,\C)$ für $n \in \N$ und $h \in C(D\times\Gamma,\C)$. Dann gelten:

\spacing

$(f_n)$ konv. glm. auf $\Gamma$ gegen $f$

\vspace*{-2mm}
$$\implies \displaystyle\lim_{n\to\infty} \int_\gamma f_n dz = \int_\gamma f dz$$

$\sum_{n=1}^\infty f_n$ konv. glm. auf $\Gamma$

\vspace*{-2mm}
$$\implies \displaystyle\sum_{n=1}^\infty \int_\gamma f_n dz = \int_\gamma \displaystyle\sum_{n=1}^\infty f_n dz$$

Abbildung $H : z \mapsto \int_\Gamma h(z,w) dw \in C(D,\C)$

\spacing

$z \mapsto h(z,w) \in H(D)$ mit $\frac{\partial}{\partial z} h \in C(D \times \Gamma, \C)$

\vspace*{-2mm}
$$\implies \frac{d}{dz} \int_\gamma h(z,w) dw = \int_\gamma \frac{\partial}{\partial z} h(z,w) dw$$

d.h. $H$ ist holomorph mit dieser Ableitung.

\subsection*{Konstant auf Gebieten}

Sei $D \subset \C$ ein Gebiet, $f \in H(D)$ und $f'=0$ auf $D$. Dann ist $f$ konstant.
