% Borel Sigma-Algebra Kürzel
\newcommand{\A}{\mathcal{A}}
\newcommand{\B}{\mathcal{B}}
\newcommand{\C}{\mathcal{C}}
\newcommand{\E}{\mathcal{E}}
\newcommand{\F}{\mathcal{F}}
\newcommand{\J}{\mathcal{J}}
\renewcommand{\L}{\mathcal{L}}

% indicator function symbol
\newcommand{\1}{\mathbbm{1}}

\section*{Nützliches aus der Mengenlehre}

\subsection*{De Morgansche Regeln}

Sei $\B$ ein Mengensystem.

$$\left(\bigcup_{B\in \B} B \right)^c = \bigcap_{B\in \B} B^c \hspace*{8mm} \left(\bigcap_{B\in \B} \right)^c = \bigcup_{B\in \B} B^c$$

\subsection*{Mengen-Ring}

Ein Mengensystem $\A$ ist ein Ring gdw. $\forall A, B \in \A$:

\begin{enumerate}[label=(\alph*)]
	\item $\emptyset \in \A$
	\item $B\setminus A \in \A$
	\item $A \cup B \in \A$
\end{enumerate}

\section*{$\sigma$-Algebren}

Ein Mengensystem $\A \subseteq \powerset{X}$ ist $\sigma$-Algebra auf der nichtleeren Menge $X$ gdw.:

\begin{enumerate}[label=(\alph*)]
	\item $X \in \A$
	\item $A \in \A \Rightarrow A^c := X\setminus A \in \A$
	\item $\forall j \in \N : A_j \in \A \Rightarrow \bigcup_{j\in \N} A_j \in \A$
\end{enumerate}

\subsection*{Eigenschaften von $\sigma$-Algebren}

Seien $\A$ eine $\sigma$-Algebra auf $X$, $n \in \N$, $\forall j \in \N : A_j \in \A$, dann ist $\A$ nach den folgenden Eigenschaften abgeschlossen unter abzählbaren Mengenoperationen:

\begin{enumerate}[label=(\alph*)]
	\item $\emptyset = X^c \in \A$
	\item $A_1 \bigcup \cdots \bigcup A_n \in \A$
	\item $A_1 \bigcap \cdots \bigcap A_n \in \A$
	\item $\bigcap_{j\in \N} A_j \in \A$
	\item $A_1 \setminus A_2 := A_1 \bigcap A_2^c \in \A$
\end{enumerate}

\subsection*{Erzeugte $\sigma$-Algebren}

Die durch das nichtleere Mengensystem $\E \subseteq \powerset{X}$ auf $X$ erzeugte $\sigma$-Algebra ist wie folgt definiert:

\vspace*{-4mm}
$$\sigma(\E) := \bigcap\{ \A \subseteq \powerset{X} | \A \text{ ist } \sigma \text{-Algebra}, \E \subseteq \A \}$$

Der Erzeuger $\E$ ist hierbei allg. nicht eindeutig.

\subsubsection*{Eigenschaften erzeugter $\sigma$-Algebren}

Sei $\emptyset \neq \E \subseteq \powerset{X}$, dann gilt:

\begin{enumerate}[label=(\alph*)]
	\item $\A$ ist $\sigma$-Algebra $\land$ $\E \subseteq \A \Rightarrow \E \subseteq \sigma(\E) \subseteq \A$
	\item $\sigma(\E)$ ist kleinste $\E$ enthaltende $\sigma$-Algebra.
	\item $\E$ ist $\sigma$-Algebra $\Rightarrow \E = \sigma(\E)$
	\item $\E \subseteq \E' \subseteq \powerset{X} \Rightarrow \sigma(\E) \subseteq \sigma(\E')$
\end{enumerate}

\subsection*{Borelsche $\sigma$-Algebra}

Sei $X$ ein metrischer Raum und $\mathcal{O}(X)$ das System der in $X$ offenen Mengen, dann ist $\B(X) := \sigma(\mathcal{O}(X))$ die Borelsche $\sigma$-Algebra auf $X$.

Im Speziellen wird $\B_m := \B(\R^m)$ gesetzt.

$\B_m$ enthält insb. alle offenen und abgeschlossenen Mengen in $\R^m$ sowie deren abzählbaren Vereinigungen und Durchschnitte.

\subsubsection*{Charakterisierung}

\vspace*{-4mm}
\begin{align*}
	\B_m &= \sigma(\{(a, b) | a, b \in \mathbb{Q}^m, a \leq b\}) \\
	              &= \sigma(\{(a, b] | a, b \in \mathbb{Q}^m, a \leq b\})
\end{align*}

\section*{Maße auf $\sigma$-Algebren}

Sei $\A$ eine $\sigma$-Algebra auf $X$.

$\mu : \A \to [0, \infty]$ ist positives Maß auf $\A$ gdw.:

\begin{enumerate}[label=(\alph*)]
	\item $\mu$ wohldefiniert und nichtnegativ
	\item $\mu(\emptyset) = 0$
	\item $\forall \text{ disjunkte } \{A_j | j \in \N\} \subseteq \A :\\ \hspace*{4mm} \mu(\dot\bigcup_{j\in \N} A_j) = \sum_{j\in \N} \mu(A_j)$
\end{enumerate}

\subsection*{Maßraum}

Ein Tripel $(X, \A, \mu)$ ist Maßraum. Ein endlicher Maßraum erfüllt zusätzlich $\mu(X) < \infty$.

Ein Wahrscheinlichkeitsmaß erfüllt $\mu(X) = 1$.

\subsection*{Punkt- / Diracmaß}

Für fest gewählte $\A = \powerset{X}$, $x \in X$ ist ein Wahrscheinlichkeitsmaß für $A \subseteq X$ definiert:

$$\delta_x(A) := \begin{cases}
	1 & x \in A \\
	0 & x \notin A
\end{cases}$$

Dieses wird Punkt- / Diracmaß auf $\A$ genannt.

\subsection*{Zählmaß}

Sei $\A = \powerset{\N}$ und $\forall j \in \N : p_j \in [0, \infty]$ fest gewählt.

$\mu(A) := \sum_{j\in A} p_j$ für $A \subseteq \N$ ist Maß auf $\powerset{\N}$.

Gilt zusätzlich $\forall j \in \N : p_j = 1$ so heißt $\mu$ Zählmaß.

\subsection*{Eigenschaften von Maßen}

Sei $(X, \A, \mu)$ Maßraum und $A, B, A_j \in \A$ für $j \in \N$.

\begin{description}[leftmargin=!,labelwidth=26mm]
	\item[Monotonie] $A \subseteq B \Rightarrow \mu(A) \leq \mu(B)$
	\item[$\sigma$-Subadditivität] $\mu(\dot\bigcup_{j\in \N} A_j) \leq \sum_{j\in \N} \mu(A_j)$
	\item[Stetigkeit (unten)] $A_j \uparrow \Rightarrow \displaystyle\lim_{j\to \infty} \mu(A_j) = \mu(\bigcup_{j\in \N} A_j)$
	\item[Stetigkeit (oben)] $A_j \downarrow \land \hspace*{1mm} \mu(A_1) < \infty \\ \hspace*{4mm}\Rightarrow \displaystyle\lim_{j\to \infty} \mu(A_j) = \mu(\bigcap_{j\in \N} A_j)$
\end{description}

Für $\mu(A) < \infty$ folgt $\mu(B\setminus A) = \mu(B) - \mu(A)$.

Für endliche Maße gilt insb. $\mu(A^c) = \mu(X) - \mu(A)$.

\subsection*{Prämaß}

Eine Abb. $f : \A \to [0, \infty)$ ist ein Prämaß auf Ring $\A$ gdw.:

\begin{enumerate}[label=(\alph*)]
	\item $\mu(\emptyset) = 0$
	\item $\{A_j | j \in \N\} \subseteq \A$ disjunkt und $A = \bigcup_{j\in \N} A_j \in \A \Rightarrow \mu(A) = \sum_{j\in \N} \mu(A_j)$
\end{enumerate}

\section*{Lebesguemaß}

\subsection*{System der Intervalle}

Sei $I = (a, b] \subseteq \R^m$ für $a, b \in \R^m$ mit $a \leq b$, dann wird das System von Intervallen $\J_m$ definiert:

$\lambda(I) = \lambda_m(I) := (b_1 - a_1) \cdot \hdots \cdot (b_m - a_m)$

\subsection*{Ring der Figuren}

$$\F_m = \left\{ A = \bigcup_{j=1}^n I_j | I_j \in \J_m, n \in \N \right\}$$

\subsubsection*{Eigenschaften des Ring der Figuren}

Seien $I_1, I_2 \in \J_m$:

\begin{enumerate}[label=(\alph*)]
	\item $\sigma(\F_m) = \B_m$
	\item $I_1 \cap I_2 \in \J_m$
	\item $I_1 \setminus I_2 \in \F_m$ sowie endliche Vereinigung disjunkter Intervalle aus $\J_m$
	\item $\forall A \in \F_m: A$ ist endliche Vereinigung disjunkter Intervalle aus $\J_m$
	\item $\F_m$ ist Ring
\end{enumerate}

\section*{Messbare Funktionen}

Sei $\A$ eine $\sigma$-Algebra auf $X \neq \emptyset$ und $\B$ eine $\sigma$-Algebra auf $Y \neq \emptyset$ sowie $f : X \to Y$ Funktion.

$f$ heißt ($\A$-$\B$-)messbar gdw. $\forall B \in \B : f^{-1}(B) \in \A$

\subsection*{Borel-Messbarkeit}

Seien $X, Y$ metrische Räume.

Die Funktion $f : X \to Y$ heißt Borel-messbar, wenn sie $\B(X)$-$\B(Y)$-messbar ist.

\subsection*{Eigenschaften Borel-messbarer Fkt.}

Seien $\A, \B, \C$ $\sigma$-Algebren auf $X, Y, Z \neq \emptyset$.

\begin{enumerate}[label=(\alph*)]
	\item $f : X \to Y$ ist $\A$-$\B$-mb., $g : Y \to Z$ ist $\B$-$\C$-mb. $\Rightarrow g \circ f : X \to Z$ ist $\A$-$\C$-mb.
	\item $\emptyset \neq \E \subseteq \powerset{Y}$, $\B = \sigma(\E)$, $f: X \to Y$ dann ist $f$ messbar gdw. $\forall E \in \E : f^{-1}(E) \in \A$
	\item $X, Y$ metrische Räume, $f : X \to Y$ stetig $\Rightarrow f$ ist Borel-messbar
	\item $f : X \to \R^m$ ist $\A$-$\B_m$-mb. gdw. $f_i : X \to \R$ ist $\A$-$\B_1$-mb.
	\item $f, g$ sind $\A$-$\B_1$-mb. und $\alpha, \beta \in \R \Rightarrow fg : X \to \R$ und $\frac{1}{f} : \{x \in X | f(x) \neq 0\} \to \R$ mb.
	\item $f : X \to \R^m$ ist $\A$-$\B_m$-mb. $\\\Rightarrow g : X \to \R; x \mapsto |f(x)|_2$ ist $\A$-$\B_1$-mb.
	\item{
		$X = W \dot\cup Z$ mit $\emptyset \neq W, Z \in \A$, $f : W \to Y$ ist $\A_W$-$\B$-mb., $g : Z \to Y$ ist $\A_Z$-$\B$-mb. $\Rightarrow h(x) = \begin{cases}
			f(x) & x \in W \\
			g(x) & x \in Z
		\end{cases}$ ist $\A$-$\B$-mb.
	}
	\item Stückw. stg. $f: [a,b] \to \R$ sind Borel-mb.
	\item Für diffbare $f : [a,b] \to \R$ ist $f' : [a,b] \to \R$ $\B([a,b])-\B_1$-mb.
\end{enumerate}

\subsection*{Borel-Messbarkeit in $\overline \R$}

Für $f, g : X \to \overline \R$ seien für $=, \neq, \leq, <$ usw.:

\vspace{-4mm}
\begin{align*}
	\{ f = g \} &= \{ x \in X | f(x) = g(x) \}\\
	\{ f \leq g \} &= \{ x \in X | f(x) \leq g(x) \}
\end{align*}

Sei die Borelsche $\sigma$-Algebra $\overline \B_1$ auf $\overline \R$ definiert als: $\overline \B_1 := \{ B \cup E | B \in \B_1, E \subseteq \{+\infty, -\infty\} \}$

Weiterhin gilt: $\overline \B_1 = \sigma(\{ [-\infty,a] | a \in \Q \})$

Äquivalent sind für $\leq, <, \geq, >$:

\begin{enumerate}[label=(\alph*)]
	\item $f: X \to \overline \R$ ist messbar
	\item $\forall a \in \Q : \{f \leq a\} \in \B(X)$
\end{enumerate}

\subsection*{Konvergenzeigenschaften Borel-mb. Fkt.}

Seien $f_n : X \to \overline \R$ für alle $n \in \N$ $\A-\overline \B_1$-messbar

\vspace{-4mm}
$$\Rightarrow \sup_{n \in \N} f_n, \inf_{n \in \N} f_n, \varliminf_{n \to \infty} f_n, \varlimsup_{n \to \infty} f_n \ \ \A-\overline \B_1 \text{-messbar}$$

\vspace{-4mm}
$$\forall x \in X : \lim_{n \to \infty} f_n(x) \in \overline \R \Rightarrow \lim_{n \to \infty} f_n \text{ ist } \A-\overline \B_1-\text{mb.}$$

$f : [a,b] \to \R$ diffbar. $\Rightarrow f'$ ist $\B([a,b])$-$\B_1$-mb.

\spacing

Sei $f : X \to \overline \R$, $X_j \in \A$ mit $X_j \uparrow$, $\cup_{j \in \N} X_j = X$ s.d. $\forall j \in \N : \restrictedto{f}{X_j} : X_j \to \overline \R$ ist $\A_{X_j}$-$\overline\B_1$-mb.

$\Rightarrow f$ ist $\A$-$\overline\B_1$-mb.

\subsection*{Positiv- und Negativteil einer Funktion}

Für $f : X \to \overline \R$: $f_+ = \max\{f,0\} : X \to [0, \infty]$

\hspace{20.1mm} $f_- = \max\{-f,0\} : X \to [0, \infty]$

Es gelten $f = f_+ - f_-$, $|f| = f_+ + f_-$

$f$ ist $\A$-$\overline\B_1$-mb. gdw. $f_+$ und $f_-$ $\A$-$\overline\B_+$-mb. sind.

Dann ist auch $|f| : x \mapsto |f(x)|$ $\A$-$\overline\B_+$-mb.

\subsection*{Einfache Funktionen}

Messbare $f : X \to \R$ heißt einfach, wenn sie nur endlich viele Werte annimmt. Für $A_j := f^{-1}(\{y_j\}) \in \A$ ist $f = \sum_{j=1}^n y_j \1_{A_j}$  die Normalform von $f$.

Sei $f : X \to \overline R$ messbar, dann gelten:

\begin{enumerate}[label=(\alph*)]
	\item $\exists \text{ einfache } f_n : X \to \R$ mit $\lim_{n \to \infty} f_n = f$ punktweise und $\forall n \in \N : |f_n| \leq |f|$
	\item Für beschränkte $f$ gilt (a) mit glm. Konv.
	\item Für $f \geq 0$ gilt (a) mit $f_n \leq f_{n+1} \leq f$
\end{enumerate}

$f : X \to \overline\R$ ist $\A$-$\overline\B_1$-mb. gdw. einfache Fkt. $f_n : X \to \R$ ex., welche punktweise gegen $f$ konv.

\section*{Lebesgue-Integral}

\subsection*{Integral für nichtnegative einfache Fkt.}

$$\int f d\mu = \int_X f(x) d\mu(x) := \sum_{j=1}^n y_j \mu(A_j) \in [0, \infty]$$

\subsection*{Integral für nichtnegative Funktionen}

Sei $f : X \to [0, \infty]$ $\A$-$\overline\B_+$-mb. und $f_n$ mit $\lim_{n \to \infty} f_n(x) = f(x)$ gegeben:

\vspace{-4mm}
\begin{align*}
	\int_X f(x) d\mu(x) :&= \lim_{n \to \infty} \int_X f_n(x) d\mu(x) \\
	                &= \sup_{n \in \N} \int_X f_n(x) d\mu(x) \in [0,\infty]
\end{align*}

Grundlegende Integraleigenschaften sind erfüllt.

\subsection*{Monotone Konvergenz}

Sei $(X, \A, \mu)$ ein Maßraum, $f_n : X \to [0,\infty]$ $\A$-$\overline\B_+$-mb. und $f_n \leq f_{n+1}$. Dann ist $f : X \to [0,\infty]$ $\A$-$\overline\B_+$-mb. und es gilt:

\vspace{-4mm}
\begin{align*}
	\int_X f(x) d\mu(x) &= \int_X \lim_{n \to \infty} f_n(x) d\mu(x)\\
	               &= \lim_{n \to \infty} \int_X f_n(x) d\mu(x)
\end{align*}

Dies gilt nicht ohne Monotonie oder für eine fallende Folge $(f_n)_{n \in \N}$.

\subsubsection*{Summen messbarer Funktionen}

Seien $f_j : X \to [0,\infty]$ $\A$-$\overline\B_+$-mb. für $\forall j \in \N$. Dann ist auch $\sum_{j=1}^\infty f_j$ $\A$-$\overline\B_+$-messbar und:

$$\int_X \sum_{j=1}^\infty f_j(x) d\mu(x) = \sum_{j=1}^\infty \int_X f_j(x) d\mu(x)$$

\subsection*{Integral für $\overline\R$-wertige Funktionen}

Sei $f : X \to \overline\R$ eine $\A$-$\overline\B_1$-mb. Funktion. Dann sind auch $f_+$ und $f_-$ mb. $f$ ist Lebesgue-integrierbar, wenn:

\vspace{-4mm}
$$\int_X f_+(x) d\mu(x) < \infty \text{ und } \int_X f_-(x) d\mu(x) < \infty$$

Das Lebesgue-Integral ist dann definiert durch:

$$\int f d\mu = \int_X f_+(x) d\mu(x) - \int_X f_-(x) d\mu(x)$$

$\L^1(X,\A,\mu) = \L^1(\mu) = \L^1(X) := \{ f : X \to \R | f \text{ ib.}\}$

\subsubsection*{Charakterisierung der Integrierbarkeit}

Für $\A-\overline\B_1$-mb. Fkt. $f : X \to \overline\R$ sind äquivalent:

\begin{enumerate}[label=(\alph*)]
	\item $f$ ist integrierbar
	\item Es ex. integrierbare Fkt. $u, b : X \to [0,\infty]$ mit $f=u-v$ wobei $\not\exists x \in X : u(x)=v(x)=\infty$
	\item Es ex. ib. Fkt. $g : X \to [0,\infty]$ mit $|f| \leq g$
	\item $|f| : X \to [0,\infty]$ ist ib d.h. $\int_X |f| d\mu < \infty$
\end{enumerate}

\subsubsection*{Eigenschaften des Integrals}

\begin{enumerate}[label=(\alph*)]
	\item $\int_Y \restrictedto{f}{Y}(x) d\mu_Y(x) = \int_X \1_Y(x) f(x) d\mu(x)$
	\item $\int_X \alpha f(x) d\mu(x) = \alpha \int_X f(x) d\mu(x)$
	\item $\int_{X_0} (f + g) d\mu = \int_X f d\mu + \int_X g d\mu$
	\item $\max\{f,g\}$ und $\min\{f,g\} : X \to \overline\R$ sind ib.
	\item Sei $f \leq g$. Dann ist $\int_X f d\mu \leq \int_X g d\mu$
	\item $|\int_X f d\mu| \leq \int_X |f| d\mu$
	\item Sei $h : X \to \R$ mb. und beschränkt mit $\mu(\{h \neq 0\}) < \infty$. Dann ist $h$ integrierbar und: $|\int_X h d\mu| \leq \|h\|_\infty \mu(\{h \neq 0\})$
	\item Sei $A \in \A$ mit $\mu(A) = 0$ und $h : X \to \overline\R$ $\A$-$\overline\B_1$-mb. Dann ist $\1_A h : X \to \overline\R$ ib. und $\int_A h d\mu = 0$
\end{enumerate}

$\L^1(\mu)$ ist Vektorraum und das Integral ist eine lineare Abbildung von $\L^1(\mu)$ nach $\R$.

\spacing

Sei $f$ einfach mit $f := \sum_{j=1}^n y_j \1_{B_j}$ mit $y_j \in \R$, $B_j \in \A$ und $\mu(B_j) < \infty$. Dann: $\int_X f d\mu = \sum_{j=1}^n y_j \mu(B_j)$

\subsubsection*{Übereinstimmung Riemann-Integral}

Sei $f : [a,b] \to \R$ stckw. stetig. Dann ist $f$ Lebesgue- und Riemann-integrierbar, die beiden Integrale stimmen überein.

Der Hauptsatz der Differential- und Integralrechnung gilt auch für das Lebesgue-Integral.

\subsection*{Nullmengen}

Mengen $N \in \A$ mit $\mu(N) = 0$ heißen Nullmengen.

Die rationalen Zahlen $\Q$ sind eine $\lambda_1$-Nullmenge, Hyperebenen in $\R^m$ sind $\lambda_m$-Nullmengen.

\subsubsection*{Eigenschaften von Nullmengen}

\begin{enumerate}[label=(\alph*)]
	\item $M, N \in \A$, $M \subseteq N$ ist $\mu$-Nullmenge \\ $\Rightarrow$ $M$ ist $\mu$-Nullmenge
	\item $\forall j \in \N : N_j$ ist Nullmenge \\ $\Rightarrow N = \cup_{j \in \N} N_j$ ist Nullmenge
	\item Überabzählbare Vereinigungen von Nullmengen können das Maß $\infty$ besitzen
	\item Borelmenge $A$ ist $\lambda_m$-Nullmenge gdw. für $\forall \epsilon > 0$  offene Intervalle $I_j$ existieren s.d. $A \subseteq \cup_{j \in \N} I_j$ und $\sum_{j=1}^\infty \lambda_m(I_j) \leq \epsilon$
	\item $\lambda_m$-Nullmenge hat keinen inneren Punkt
\end{enumerate}

Ein Maßraum $(X,\A,\mu)$ heißt vollständig, wenn $\forall M \subseteq$ Nullmenge $N : M \in \A$.

\spacing

$\tilde \A := \{ \tilde A = A \cup M | A \in \A, M \subseteq N$ für eine $\mu$-Nullmenge $N\}$ ergibt Vervollständigung $(X,\tilde\A,\tilde\mu)$ eines beliebigen Maßraum $(X,\A,\mu)$.

\subsubsection*{Definition fast überall}

Eine Eigenschaft $E$ besteht für fast alle $x \in X$ oder fast überall. wenn es Nullmengen $N$ gibt s.d. $E$ für alle $x \in X \setminus N$ gilt.

\spacing

Sei $f : X \to \overline\R$ ib. Dann ist $\{|f|=\infty\}$ eine Nullmenge, $f$ ist also fast überall endlich.

\subsection*{Lemma von Fatou}

Sei $f_n : X \to [0,\infty]$ für alle $n \in \N$ mb. Dann:

\vspace{-2mm}
$$\int_X \liminf_{n \to \infty} f_n d\mu \leq \liminf_{n \to \infty} \int_X f_n d\mu$$

Konvergiert $f_n$ f.ü. gegen mb. $f : X \to [0,\infty]$:

\vspace{-2mm}
$$\int_X f d\mu \leq \liminf_{n \to \infty} \int_X f_n d\mu$$

\subsection*{Majorisierte Konvergenz (Lebesgue)}

Sei $f, f_n : X \to \overline\R$ messbar und $g : X \to [0,\infty]$ integrierbar. Konvergiere $(f_n)$ in $\overline\R$ f.ü. gegen $f$ und $\forall n \in \N : |f_n| \leq g$ f.ü.

Dann sind $f$ und $f_n$ für $\forall n \in \N$ integrierbar und:

\vspace{-4mm}
\begin{align*}
	\lim_{n \to \infty} \int_X f_n d\mu &= \int_X f d\mu \\
	|\int_X f_n d\mu - \int_X f d\mu| &\leq \int_{X \setminus N} |f_n - f| d\mu \to 0
\end{align*}

Mit $N := \{|f| = \infty\} \cup \cup_{n \in \N} \{|f_n| = \infty\}$ Nullmenge.

\subsection*{Stetigkeitssatz}

Sei $M$ metrischer Raum, $t_0 \in M$, $(X,\A,\mu)$ Maßraum und $f : M \times X \to \R$ erfülle:

\begin{enumerate}[label=(\alph*)]
	\item $\forall t \in M$ ist $x \mapsto f(t,x)$ messbar
	\item Es ex. ib. $g : X \to [0,\infty]$ und Nullmengen $N_t$ für $t \in M$, dass $|f(t,x)| \leq g(x)$ für alle $t \in M$ und $x \in X \setminus N_t$ gilt
	\item Es ex. Nullmenge $N$ s.d. $t \mapsto f(t,x)$ für $x \in X \setminus N$ bei $t_0$ stetig ist
\end{enumerate}

Dann ist $\forall t \in M$ die Fkt. $X \to \R; x \mapsto f(t,x)$ ib. und es gilt:

\vspace{-2mm}
$$\lim_{t \to t_0} \int_X f(t,x) d\mu(x) = \int_X f(t_0,x) d\mu(x)$$

\subsection*{Differentiationssatz}

Sei $U \subseteq \R^k$ offen, $j \in \{1,\cdots,k\}$, $(X,\A,\mu)$ Maßraum und $f : U \times X \to \R$ erfülle:

\begin{enumerate}[label=(\alph*)]
	\item $\forall t \in U : x \mapsto f(t,x)$ ist integrierbar
	\item $\exists$ Nullmenge $N_1 \forall t \in U \land x \in X \setminus N_1 : \frac{\partial}{\partial t_j} f(t,x)$ existiert
	\item $\exists$ Nullmenge $N_2$ und ib. $g : X \to [0,\infty] : \forall x \in X \setminus N_2, t \in U : |\frac{\partial f}{\partial t_j} (t,x)| \leq g(x)$
\end{enumerate}

Dann ist $\forall t \in U$ die Abbildung $x \mapsto \frac{\partial}{\partial t_j} f(t,x)$ integierbar und es ex. die partielle Ableitung:

$$\frac{\partial}{\partial t_j} \int_X f(t,x) dx = \int_X \frac{\partial f}{\partial t_j} (t,x) dx$$

\section*{Iterierte Integrale}

Darstellung von Integralen auf $\R^m$ als Komposition von Integralen auf $\R^k$ und $\R^l$ mit $m = k + l$.

Für $C \subseteq \R^m$ sind die Schnitte definiert:

\vspace{-4mm}
\begin{align*}
	C_y &= \{ x \in \R^k | (x,y) \in C \} &\text{ für feste } y \in \R^l\\
	C^x &= \{ y \in \R^l | (x,y) \in C \} &\text{ für feste } x \in \R^k
\end{align*}

\subsection*{Prinzip des Cavalieri}

Für beliebige $C \in \B_m$ gilt:

$$\lambda_m(C) = \int_{\R^k} \lambda_l(C^x) dx = \int_{\R^l} \lambda_k(C_y) dy$$

Daraus folgt:

\vspace{-4mm}
\begin{align*}
\int_{\R^m} \1_C(z) dz &= \int_{\R^k} \left( \int_{\R^l} \1_C(x,y) dy \right) dx\\
                         &= \int_{\R^l} \left( \int_{\R^k} \1_C(x,y) dx \right) dy
\end{align*}

\subsection*{Satz von Tonelli}

Sei $f : \R^m \to [0,\infty]$ messbar. Dann:

\vspace{-4mm}
\begin{align*}
	\int_{\R^m} f(z) dz &= \int_{\R^k} \left( \int_{\R^l} f(x,y) dy \right) dx\\
	             &= \int_{\R^l} \left( \int_{\R^k} f(x,y) dx \right) dy
\end{align*}

\subsection*{Satz von Fubini}

Sei $f : \R^m \to \overline\R$ integrierbar. Dann ex. Nullmengen $M \in \B_k$ und $N \in \B_l$ s.d. $f^x : \R^l \to \overline\R$ für alle $x \in \R^k \setminus M$ und $f_y : \R^k \to \overline\R$ für alle $y \in \R^l \setminus N$ integrierbar sind.

Dann sind $x \mapsto \int_{\R^l} f(x,y) dy$ und $y \mapsto \int_{\R^k} f(x,y) dx$ integrierbar und es gilt der Satz von Tonelli.

\subsection*{Transformationssatz}

Sei $U \subseteq \R^m$ offen, $\phi \in C^1(U,\R^m)$ injektiv und $A \in \B_m$ mit $A \subseteq U$ s.d. $A^\circ \neq \emptyset$ und $A \setminus A^\circ$ Nullmenge ist. Sei $\phi(A) \in \B_m$ und $\forall x \in A^\circ : \det \phi'(x) \neq 0$.

Dann ist $\phi(A^\circ)$ offen, $\phi : A^\circ \to \phi(A^\circ)$ Diffeomorphismus und $\phi(A) \setminus \phi(A^\circ)$ Nullmenge. Weiter:

\begin{enumerate}[label=(\alph*)]
	\item Sei $f: \phi(A) \to [0,\infty]$ messbar. Dann: \vspace{-2mm} $$\int_{\phi(A)} f(y) dy = \int_A f(\phi(x))|\det \phi'(x)| dx$$
	\item Sei $f : \phi(A) \to \overline\R$ messbar. Dann ist $f$ auf $\phi(A)$ ib. gdw. $x \mapsto f(\phi(x))|\det(\phi'(x))|$ auf $A$ integrierbar ist. Es gilt dann auch (a).
\end{enumerate}

\section*{Komplexe Integrale}

Der metrische Raum $\mathbb{C}$ ist homöomorph zu $\R^2$, $\B(\mathbb{C})$ wird mit $\B_2$ identifiziert.

$f : X \to \mathbb{C}$ ist $\A$-$\B(\mathbb{C})$-mb. gdw. $\text{Re} f, \text{Im} f : X \to \R$ $\A$-$\B_1$-messbar sind.

Für die Integrierbarkeit von mb. $f : X \to \mathbb{C}$ gilt:

$|f| : X \to [0,\infty)$ ib. $\Leftrightarrow \text{Re} f, \text{Im} f : X \to \R$ ib.

$$\int_X f d\mu := \int_X \text{Re} f d\mu + i \int_X \text{Im} f d\mu$$

\section*{Differentialgeometrie}

\subsection*{$C^1$-Hyperflächen}

Eine Menge $M \subseteq \R^m$ ist eingebettete $C^1$-Hyperfläche, wenn $\forall x \in M$ offene Mengen $U, V \subseteq \R^m$ und Diffeomorphismus $\psi : V \to U$ existieren s.d. $x \in V$ und $\psi(V \cap M) = U \cap (\R^{m-1} \times \{0\})$ gilt.

Die Abbildung $\psi$ heißt dann Karte.

\subsection*{$C^k$-Hyperflächen}

Liegt die Karte $\psi$ einer $C^1$-Hyperfläche $M$ in $C^k(CV,\R^m)$, dann ist $M$ eine $C^k$-Hyperfläche.

\subsection*{Dünnsinguläre $C^k$-Hyperflächen}

Eine Borelmenge $M \subseteq \R^m$ ist \emph{dünnsinguläre} $C^k$-Hyperfläche gdw. $C^k$-Hyperfläche $M_r \subseteq M$ mit $\overline M_r = M$ und $k \in \N \cup \{\infty\}$ existiert s.d. $N = M \setminus M_r$ eine $(m-1)$-dimensionale Nullmenge ist.

\subsection*{Gramsche Determinante}

Sei $F : U \to W$ eine Parametrisierung. Dann ist

\vspace{-2mm}
$$g_F(t) = \det(F'(t)^TF'(t))$$

die \emph{Gramsche Determinante} von $F$.

\spacing

Für $m = 3$ gilt insbesondere:

\vspace{-4mm}
\begin{align*}
g_F(t) &= |\partial_1F(t) \times \partial_2F(t)|_2^2\\
       &= \left|\begin{pmatrix}
	\partial_1 F_2(t) \partial_2 F_3(t) - \partial_1 F_3(t) \partial_2 F_2(t) \\
	\partial_1 F_3(t) \partial_2 F_1(t) - \partial_1 F_1(t) \partial_2 F_3(t) \\
	\partial_1 F_1(t) \partial_2 F_2(t) - \partial_1 F_2(t) \partial_2 F_1(t)
\end{pmatrix}\right|_2^2
\end{align*}

Im Graphenfall $F(t) = (t,h(t))$ für $t \in U$, $U \subseteq \R^{m-1}$ offen und $h \in C^1(U,\R)$ gilt:

$\sqrt{g_F(t)} = \sqrt{1+|\nabla h(t)|_2^2}$

\subsection*{Oberflächenintegral}

Sei $F : U \to W$ eine Parametrisierung, $M_0 = F(U) \subseteq \R^m$ ein offenes Flächenstück. Sei weiter $f : M_0 \to \overline\R$ messbar und nichtnegativ oder die Funktion $g := f \circ F \sqrt{g_F}$ integrierbar. Dann:

\vspace{-4mm}
$$\int_{M_0} f d\sigma = \int_{M_0} f(x) d\sigma(x) := \int_U f(F(t))\sqrt{g_F(t)} dt$$

\subsubsection*{Oberflächenmaß}

Sei $B \in \B(M_0)$ dann ist $\1_B$ messbar und $F^{-1}(B) \in \B(U)$. Dann ist das Oberflächenmaß definiert:

\vspace{-4mm}
\begin{align*}
\sigma(B) := \int_{M_0} \1_B d\sigma &= \int_U \1_B(F(t)) \sqrt{g_F(t)} dt\\
                             &= \int_{F^{-1}(B)} \sqrt{g_F(t)} dt
\end{align*}

\subsection*{Divergenzsatz von Gauß}

Sei $D \subseteq \R^m$ offen und beschränkt mit dünnsingulärem $C^1$-Rand, $f \in C(D,\R^m) \cap C_b^1(D,\R^m)$ und $(f|\nu) \in \L^1(\partial D,\sigma)$. Dann:

$$\int_D div f(x) dx = \int_{\partial D} (f(x)|\nu(x)) d\sigma(x)$$

Mit $\text{div} f(x) := \text{spur} f'(x) = \partial_1 f_1(x) + \cdots + \partial_m f_m(x)$ und $\nu$ ist äußere Einheitsnormale.

\subsection*{Satz von Stokes in $\R^3$}

Für $f \in C^1(D,\R^3)$ ist die Rotation definiert:

$$\text{rot} f(x) = \begin{pmatrix}
	\partial_2 f_3(x) - \partial_3 f_2(x) \\
	\partial_3 f_1(x) - \partial_1 f_3(x) \\
	\partial_1 f_2(x) - \partial_2 f_1(x)
\end{pmatrix}$$

Dann gilt mit der äußeren Einheitsnormalen $n$:

$$\int_M (\text{rot} f(x) | n(x)) d\mu(x) = \int_{\partial_2 M} f \cdot dx$$

Dabei ist das \emph{Kurvenintegral zweiter Art} geg. als:

$$\int_{\partial_2 M} f \cdot dx = \int_a^b (f(\varphi(\tau))|\varphi'(\tau)) d\tau$$

\section*{Lebesguesche Räume}

Für messbare $f : X \to \overline\R$:

\vspace{-4mm}
\begin{align*}
\|f\|_p &= \left(\int_X |f|^p d\mu\right)^\frac{1}{p} \text{ für } p \in [1,\infty)\\
\|f\|_\infty &= \text{esssup}_{x \in X} \|f(x)\|\\
      &= \inf\left\{ c > 0 | \exists \text{ NM } N_c : \forall x \in X \setminus N_c : |f(x)| \leq c\right\}
\end{align*}

\subsection*{$\L^p$-Räume}

\vspace{-2mm}
\begin{align*}
\L^p(X,\A,\mu) &:= \{ f : X \to \R | f \text{ mb.}, \|f\|_p < \infty\} \\
\L_\mathbb{C}^p(X,\A,\mu) &:= \{ f : X \to \mathbb{C} | f \text{ mb.}, |f| \in \L^p(\mu) \}
\end{align*}

$\L^p(\mu)$ ist für $p \in [1,\infty]$ ein $\R$-Vektorraum. Zusätzlich ist $f \to \|f\|_p$ homogen und erfüllt die $\Delta$-UGL, ist jedoch bei Existenz einer $\mu$-Nullmenge $N \neq \emptyset$ nicht definit wg. $\|\1_N\|_p = 0$ und $\1_N \neq 0$.

\vspace{-4mm}
\begin{align*}
\mathcal{N} &= \{ f : X \to \R | f \text{ ist mb.}, f = 0 \text{ f.ü.}\} \\
L^p(X,\A,\mu) &= \L^p(\mu) / \mathcal{N}
\end{align*}

Der Quotientenraum $L^p(\mu)$ ist $\R$-Vektorraum mit Restklassen $\hat f = f + \mathcal{N}$ für $f \in \L^p(\mu)$. Es gilt $\hat f = \hat g$ gdw. alle $f \in \hat f$ und $g \in \hat g$ fast überall gleich sind.

\spacing

$(L^p(\mu),\|\cdot\|_p)$ ist $\forall p \in [1,\infty]$ normierter Vektorraum.

\subsubsection*{Einfache Funktionen in $\L^p$}

Sei $(X,\A,\mu)$ Maßraum und $f \in L^p(\mu)$. Dann liegt $E = \{ f \in L^p(\mu) | f \text{ ist einfach} \}$ dicht in $L^p(\mu)$, d.h:

\vspace{-4mm}
$$\forall f \in L^p(\mu), \epsilon > 0 \exists \text{ einf. } g \in L^p(\mu) : \| f - g \|_p \leq \epsilon$$

\subsection*{Hölder Ungleichung}

Sei $\frac{1}{p} + \frac{1}{p'} = 1$ mit:

\vspace{-4mm}
$$p' = \begin{cases}
	\frac{p}{p-1} & p \in (1, \infty) \\
	\infty        & p = 1 \\
	1        & p = \infty
\end{cases}$$

Dann liegt für $f \in \L^p(\mu)$, $g \in \L^{p'}(\mu)$ das Produkt $fg \in \L^1(\mu)$ und die Höldersche Ungleichung gilt:

\vspace{-4mm}
$$\left| \int_X fg d\mu \right| \leq \int_X |fg| d\mu = \|fg\|_1 \leq \|f\|_p \|g\|_{p'}$$

\subsection*{Minkowski Ungleichung}

Seien $f, g \in \L^p(\mu)$. Dann gilt $f + g \in \L^p(\mu)$ und:

\vspace{-2mm}
$$\| f + g \|_p \leq \|f\|_p + \|g\|_p$$
