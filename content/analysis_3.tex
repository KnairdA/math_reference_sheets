% Borel Sigma-Algebra Kürzel
\newcommand{\A}{\mathcal{A}}
\newcommand{\B}{\mathcal{B}}
\newcommand{\C}{\mathcal{C}}
\newcommand{\E}{\mathcal{E}}
\newcommand{\J}{\mathcal{J}}
\newcommand{\F}{\mathcal{F}}

\section*{Nützliches aus der Mengenlehre}

\subsection*{De Morgansche Regeln}

Sei $\B$ ein Mengensystem.

$$\left(\bigcup_{B\in \B} B \right)^c = \bigcap_{B\in \B} B^c \hspace*{8mm} \left(\bigcap_{B\in \B} \right)^c = \bigcup_{B\in \B} B^c$$

\subsection*{Mengen-Ring}

Ein Mengensystem $\A$ ist ein Ring gdw. $\forall A, B \in \A$:

\begin{enumerate}[label=(\alph*)]
	\item $\emptyset \in \A$
	\item $B\setminus A \in \A$
	\item $A \cup B \in \A$
\end{enumerate}

\section*{$\sigma$-Algebren}

Ein Mengensystem $\A \subseteq \powerset{X}$ ist $\sigma$-Algebra auf der nichtleeren Menge $X$ gdw.:

\begin{enumerate}[label=(\alph*)]
	\item $X \in \A$
	\item $A \in \A \Rightarrow A^c := X\setminus A \in \A$
	\item $\forall j \in \N : A_j \in \A \Rightarrow \bigcup_{j\in \N} A_j \in \A$
\end{enumerate}

\subsection*{Eigenschaften von $\sigma$-Algebren}

Seien $\A$ eine $\sigma$-Algebra auf $X$, $n \in \N$, $\forall j \in \N : A_j \in \A$, dann ist $\A$ nach den folgenden Eigenschaften abgeschlossen unter abzählbaren Mengenoperationen:

\begin{enumerate}[label=(\alph*)]
	\item $\emptyset = X^c \in \A$
	\item $A_1 \bigcup \cdots \bigcup A_n \in \A$
	\item $A_1 \bigcap \cdots \bigcap A_n \in \A$
	\item $\bigcap_{j\in \N} A_j \in \A$
	\item $A_1 \setminus A_2 := A_1 \bigcap A_2^c \in \A$
\end{enumerate}

\subsection*{Erzeugte $\sigma$-Algebren}

Die durch das nichtleere Mengensystem $\E \subseteq \powerset{X}$ auf $X$ erzeugte $\sigma$-Algebra ist wie folgt definiert:

\vspace*{-4mm}
$$\sigma(\E) := \bigcap\{ \A \subseteq \powerset{X} | \A \text{ ist } \sigma \text{-Algebra}, \E \subseteq \A \}$$

Der Erzeuger $\E$ ist hierbei allg. nicht eindeutig.

\subsubsection*{Eigenschaften erzeugter $\sigma$-Algebren}

Sei $\emptyset \neq \E \subseteq \powerset{X}$, dann gilt:

\begin{enumerate}[label=(\alph*)]
	\item $\A$ ist $\sigma$-Algebra $\land$ $\E \subseteq \A \Rightarrow \E \subseteq \sigma(\E) \subseteq \A$
	\item $\sigma(\E)$ ist kleinste $\E$ enthaltende $\sigma$-Algebra.
	\item $\E$ ist $\sigma$-Algebra $\Rightarrow \E = \sigma(\E)$
	\item $\E \subseteq \E' \subseteq \powerset{X} \Rightarrow \sigma(\E) \subseteq \sigma(\E')$
\end{enumerate}

\subsection*{Borelsche $\sigma$-Algebra}

Sei $X$ ein metrischer Raum und $\mathcal{O}(X)$ das System der in $X$ offenen Mengen, dann ist $\B(X) := \sigma(\mathcal{O}(X))$ die Borelsche $\sigma$-Algebra auf $X$.

Im Speziellen wird $\B_m := \B(\R^m)$ gesetzt.

$\B_m$ enthält insb. alle offenen und abgeschlossenen Mengen in $\R^m$ sowie deren abzählbaren Vereinigungen und Durchschnitte.

\subsubsection*{Charakterisierung}

\vspace*{-4mm}
\begin{align*}
	\B_m &= \sigma(\{(a, b) | a, b \in \mathbb{Q}^m, a \leq b\}) \\
	              &= \sigma(\{(a, b] | a, b \in \mathbb{Q}^m, a \leq b\})
\end{align*}

\section*{Maße auf $\sigma$-Algebren}

Sei $\A$ eine $\sigma$-Algebra auf $X$.

$\mu : \A \to [0, \infty]$ ist positives Maß auf $\A$ gdw.:

\begin{enumerate}[label=(\alph*)]
	\item $\mu(\emptyset) = 0$
	\item $\forall \text{ disjunkte } \{A_j | j \in \N\} \subseteq \A :\\ \hspace*{4mm} \mu(\dot\bigcup_{j\in \N} A_j) = \sum_{j\in \N} \mu(A_j)$
\end{enumerate}

\subsection*{Maßraum}

Ein Tripel $(X, \A, \mu)$ ist Maßraum. Ein endlicher Maßraum erfüllt zusätzlich $\mu(X) < \infty$.

Ein Wahrscheinlichkeitsmaß erfüllt $\mu(X) = 1$.

\subsection*{Punkt- / Diracmaß}

Für fest gewählte $\A = \powerset{X}$, $x \in X$ ist ein Wahrscheinlichkeitsmaß für $A \subseteq X$ definiert:

$$\delta_x(A) := \begin{cases}
	1 & x \in A \\
	0 & x \notin A
\end{cases}$$

Dieses wird Punkt- / Diracmaß auf $\A$ genannt.

\subsection*{Zählmaß}

Sei $\A = \powerset{\N}$ und $\forall j \in \N : p_j \in [0, \infty]$ fest gewählt.

$\mu(A) := \sum_{j\in A} p_j$ für $A \subseteq \N$ ist Maß auf $\powerset{\N}$.

Gilt zusätzlich $\forall j \in \N : p_j = 1$ so heißt $\mu$ Zählmaß.

\subsection*{Eigenschaften von Maßen}

Sei $(X, \A, \mu)$ Maßraum und $A, B, A_j \in \A$ für $j \in \N$.

\begin{description}[leftmargin=!,labelwidth=26mm]
	\item[Monotonie] $A \subseteq B \Rightarrow \mu(A) \leq \mu(B)$
	\item[$\sigma$-Subadditivität] $\mu(\dot\bigcup_{j\in \N} A_j) \leq \sum_{j\in \N} \mu(A_j)$
	\item[Stetigkeit (unten)] $A_j \uparrow \Rightarrow \displaystyle\lim_{j\to \infty} \mu(A_j) = \mu(\bigcup_{j\in \N} A_j)$
	\item[Stetigkeit (oben)] $A_j \downarrow \land \hspace*{1mm} \mu(A_1) < \infty \\ \hspace*{4mm}\Rightarrow \displaystyle\lim_{j\to \infty} \mu(A_j) = \mu(\bigcap_{j\in \N} A_j)$
\end{description}

Für $\mu(A) < \infty$ folgt $\mu(B\setminus A) = \mu(B) - \mu(A)$.

Für endliche Maße gilt insb. $\mu(A^c) = \mu(X) - \mu(A)$.

\subsection*{Prämaß}

Eine Abb. $f : \A \to [0, \infty)$ ist ein Prämaß auf Ring $\A$ gdw.:

\begin{enumerate}[label=(\alph*)]
	\item $\mu(\emptyset) = 0$
	\item $\{A_j | j \in \N\} \subseteq \A$ disjunkt und $A = \bigcup_{j\in \N} A_j \in \A \Rightarrow \mu(A) = \sum_{j\in \N} \mu(A_j)$
\end{enumerate}

\section*{Lebesguemaß}

\subsection*{System der Intervalle}

Sei $I = (a, b] \subseteq \R^m$ für $a, b \in \R^m$ mit $a \leq b$, dann wird das System von Intervallen $\J_m$ definiert:

$\lambda(I) = \lambda_m(I) := (b_1 - a_1) \cdot \hdots \cdot (b_m - a_m)$

\subsection*{Ring der Figuren}

$$\F_m = \left\{ A = \bigcup_{j=1}^n I_j | I_j \in \J_m, n \in \N \right\}$$

\subsubsection*{Eigenschaften}

Seien $I_1, I_2 \in \J_m$:

\begin{enumerate}[label=(\alph*)]
	\item $\sigma(\F_m) = \B_m$
	\item $I_1 \cap I_2 \in \J_m$
	\item $I_1 \setminus I_2 \in \F_m$ sowie endliche Vereinigung disjunkter Intervalle aus $\J_m$
	\item $\forall A \in \F_m: A$ ist endliche Vereinigung disjunkter Intervalle aus $\J_m$
	\item $\F_m$ ist Ring
\end{enumerate}

\section*{Messbare Funktionen}

Sei $\A$ eine $\sigma$-Algebra auf $X \neq \emptyset$ und $\B$ eine $\sigma$-Algebra auf $Y \neq \emptyset$ sowie $f : X \to Y$ Funktion.

$f$ heißt ($\A$-$\B$-)messbar gdw. $\forall B \in \B : f^{-1}(B) \in \A$

\subsection*{Borel-Messbarkeit}

Seien $X, Y$ metrische Räume.

Die Funktion $f : X \to Y$ heißt Borel-messbar, wenn sie $\B(X)$-$\B(Y)$-messbar ist.

\subsection*{Eigenschaften}

Seien $\A, \B, \C$ $\sigma$-Algebren auf $X, Y, Z \neq \emptyset$.

\begin{enumerate}[label=(\alph*)]
	\item $f : X \to Y$ ist $\A$-$\B$-mb., $g : Y \to Z$ ist $\B$-$\C$-mb. $\Rightarrow g \circ f : X \to Z$ ist $\A$-$\C$-mb.
	\item $\emptyset \neq \E \subseteq \powerset{Y}$, $\B = \sigma(\E)$, $f: X \to Y$ dann ist $f$ messbar gdw. $\forall E \in \E : f^{-1}(E) \in \A$
	\item $X, Y$ metrische Räume, $f : X \to Y$ stetig $\Rightarrow f$ ist Borel-messbar
	\item $f : X \to \R^m$ ist $\A$-$\B_m$-mb. gdw. $\forall i \in \{1, \dots, m\} : f_i : X \to \R$ ist $\A$-$\B_1$-mb.
	\item $f, g$ sind $\A$-$\B_1$-mb. und $\alpha, \beta \in \R \Rightarrow fg : X \to \R$ und $\frac{1}{f} : \{x \in X | f(x) \neq 0\} \to \R$ mb.
	\item $f : X \to \R^m$ ist $\A$-$\B_m$-mb. $\\\Rightarrow g : X \to \R; x \mapsto |f(x)|_2$ ist $\A$-$\B_1$-mb.
	\item{
		$X = W \dot\cup Z$ mit $\emptyset \neq W, Z \in \A$, $f : W \to Y$ ist $\A_W$-$\B$-mb., $g : Z \to Y$ ist $\A_Z$-$\B$-mb. $\Rightarrow h(x) = \begin{cases}
			f(x) & x \in W \\
			g(x) & x \in Z
		\end{cases}$ ist $\A$-$\B$-mb.
	}
	\item Stückw. stg. $f: [a,b] \to \R$ sind Borel-mb.
\end{enumerate}

\subsection*{Borel-Messbarkeit in $\overline \R$}

Für $f, g : X \to \overline \R$ seien für $=, \neq, \leq, <$ usw.:

\vspace{-4mm}
\begin{align*}
	\{ f = g \} &= \{ x \in X | f(x) = g(x) \}\\
	\{ f \leq g \} &= \{ x \in X | f(x) \leq g(x) \}
\end{align*}

Sei die Borelsche $\sigma$-Algebra $\overline \B_1$ auf $\overline \R$ definiert als: $\overline \B_1 := \{ B \cup E | B \in \B_1, E \subseteq \{+\infty, -\infty\} \}$

Weiterhin gilt: $\overline \B_1 = \sigma(\{ [-\infty,a] | a \in \Q \})$

Äquivalent sind für $\leq, <, \geq, >$:

\begin{enumerate}[label=(\alph*)]
	\item $f: X \to \overline \R$ ist messbar
	\item $\forall a \in \Q : \{f \leq a\} \in \B(X)$
\end{enumerate}

\subsection*{Konvergenzeigenschaften Borel-mb. Fkt.}

Seien $f_n : X \to \overline \R$ für alle $n \in \N$ $\A-\overline \B_1$-messbar

\vspace{-4mm}
$$\Rightarrow \sup_{n \in \N} f_n, \inf_{n \in \N} f_n, \varliminf_{n \to \infty} f_n, \varlimsup_{n \to \infty} f_n \ \ \A-\overline \B_1 \text{-messbar}$$

\vspace{-4mm}
$$\forall x \in X : \lim_{n \to \infty} f_n(x) \in \overline \R \Rightarrow \lim_{n \to \infty} f_n \text{ ist } \A-\overline \B_1-\text{mb.}$$

$f : [a,b] \to \R$ diffbar. $\Rightarrow f'$ ist $\B([a,b])$-$\B_1$-mb.

\vspace{2mm}

Sei $f : X \to \overline \R$, $X_j \in \A$ mit $X_j \uparrow$, $\cup_{j \in \N} X_j = X$ s.d. $\forall j \in \N : \restrictedto{f}{X_j} : X_j \to \overline \R$ ist $\A_{X_j}$-$\overline \B_1$-mb.

$\Rightarrow f$ ist $\A$-$\overline \B_1$-mb.
