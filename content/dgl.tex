\section*{Anfangswertprobleme}

Seien \(\J \subseteq \R\) ein Intervall, \(t_0 \in \J\) mit \(t_0 < \sup \J\), \(D \subseteq \R^m\) offen, \(f \in C(\J \times D, \R^m)\) und \(u_0 \in D\).

\vspace*{-4mm}
\begin{align*}
	u'(t)  &= f(t, u(t)), t\geq t_0, t\in \J \\
	u(t_0) &= u_0
\end{align*}

Für das Anfangswertproblem wird ein \(t_1 \in \J\) mit \(t_1 > t_0\) und eine eindeutige Lösung \(u \in C^1([t_0, t_1], \R^m)\) auf \([t_0, t_1]\) gesucht.

\subsection*{Lokale Lipschitzstetigkeit im Kontext}

Sei \(f \in C(\J \times D, \R^k)\), \(D \subseteq \R^m\) offen und es ex. alle \(\frac{\partial}{\partial x_j} f \in C(\J \times D, \R^k)\) für \(j \in \{1, \hdots, m\}\).

Dann ist \(f\) lokal Lipschitz in \(x\).

\subsection*{Picard-Lindelöf (lokal)}

Seien \(\J\) Intervall, \(D \subseteq \R^m\) offen, \(f \in C(\J \times D, \R^m)\) lokal Lipschitz in \(x\), \(u_0 \in D\), \(t_0 \in \J\) mit \(t_0 < \sup \J\). Dann gelten:

\begin{enumerate}[label=(\alph*)]
	\item \(\exists t_1 > t_0 \) mit \(t_1 \in \J\) und eind. Lsg. \(u\)  auf \([t_0, t_1]\) von \(u'(t) = f(t, u(t))\) mit \(u(t_0) = u_0\)
	\item \(u'(t) = f(t, u(t))\) mit \(u(t_0) = u_0\) besitze zwei Lsg. \(v_1\) und \(v_2\) auf \([t_0, T_1] \subseteq \J\) bzw. \([t_0, T_2] \subseteq \J\). Dann stimmen \(v_1\) und \(v_2\) auf \([t_0, T_3]\) mit \(T_3 = \min\{T_1, T_2\}\) überein.
\end{enumerate}

\subsection*{Picard-Lindelöf (maximal)}

Unter den Voraussetzungen von Picard-Lindelöf (lokal) sei \(u_0 \in D\), dann gilt:

\begin{enumerate}[label=(\alph*)]
	\item \(\exists\) max. Existenzzeit \(\overline t(u_0) \in (t_0, \sup \J]\) und eine eindeutige maximale Lösung \(u\) von \(u'(t) = f(t, u(t))\) mit \(u(t_0) = u_0\) auf \([t_0, \overline t(u_0))\)
	\item Wenn \(\overline t(u_0) < \sup \J\), dann \(\exists t_n \in (t_0, \overline t(u_0))\) mit \(\lim_{n \to \infty} t_n = \overline t(u_0)\) so, dass die Blow-Up Bedingung erfüllt ist: \(\lim_{n \to \infty} |u(t_n)|_n = \infty\) oder \(\lim_{n \to \infty} \inf_{x \in \partial D} |u(t_n) - x|_2 = 0\)
\end{enumerate}

\subsection*{Trennung der Variablen}

Sei \(u'(t)=g(t)h(u(t))\) mit \(u(t_0)=u_0\) ein AWP mit \(g \in C(\R)\), \(h \in C((a, b), \R)\), \(u_0 \in (a, b)\) und \(h(u_0) \neq 0\). \(u\) ist Lösung, wenn:

\(\forall t \in \J : u(t) \in (a, b)\), \(u \in C^1(\J, \R)\) und \(t_0 \in \J\).
\[ u \text{ ist Lösung } \Rightarrow \int_{t_0}^t g(s) ds = \int_{u_0}^{u(t)} \frac{1}{h(x)} dx \]

\subsection*{Lemma von Grönwall}

Seien \(b \in [0,\infty], \phi \in C([0,b),\R)\) und \(\alpha, \beta \geq 0\).
\[\psi(t) := \alpha + \beta \int_0^t \phi(s) ds \text{ für } t \in [0,b)\]
Weiter gelte \(\phi \leq \psi\) auf \([0,b)\). Dann gilt:
\[\forall t \in [0,b) : \phi(t) \leq \alpha \exp(\beta t)\]

\subsection*{Eindeutige Lösbarkeit}

Sei \(D = (a,b) \times \R^m\) mit \(-\infty \leq a < b \leq \infty\) und \(f : D \to \R^k\) erfülle die Vor. von Picard-Lindelöf. Gilt weiter \(\|f(t,x)\| \leq \alpha + \beta \|x\|\) für \(\alpha, \beta \geq 0\), dann ist das AWP auf \((a,b)\) eindeutig lösbar.

\section*{Autonome DGL}

Sei \(\emptyset \neq D \subseteq \R^p\) und \(g : D \to \R^p\).

Die DGL \(x'(t)=g(x(t))\) heißt \emph{autonom}.

\subsection*{Stationäre Stelle}

Sei \(x_0 \in D\) mit \(g(x_0)=0\) von \(x'(t)=g(x(t))\) für \(g : D \to \R^p\). Dann ist \(x_0\) eine stationäre Stelle.

\spacing

Sei \(g \in C(D,\R^p)\), ex. Lsg. \(x: [t_0,\infty) \to \R^p\) und \(x_0 := \lim_{t \to \infty} x(t)\). Dann ist \(x_0\) stationäre Stelle und es gilt \(x'(t) \to 0 \ (t \to \infty)\)

\subsection*{Monotone Lösung für \(p=1\)}

Sei \(D \subseteq \R\) Intervall, \(g \in C(D,\R)\) und \(x : \J \to \R\) eine Lsg. von \(x'(t)=g(x(t))\). Dann ist \(x\) monoton.

\subsection*{Fundamentalsystem}

Sei \(x'=Ax\) eine homogene lineare DGL. Ein \emph{Fundamentalsystem} ist eine Basis \(\{x_1,\dots,x_n\}\) des Lösungsraums: \[\mathcal{L} := \left\{ x \in C^1([a,b],\R^p) \middle| x=\sum_{k=1}^n a_k x_k, \ a_1,\dots,a_n \in \R \right\}\]

\subsection*{Bahn, Orbit, Trajektorie}

Seien \(\emptyset \neq D \subseteq \R^p\) offen, \(\J \subseteq \R\) und \(f : D \to \R^p\) lokal Lipschitz.

Ist \(x : \J \to \R^p\) Lsg. von \(x'(t)=f(x(t))\) so heißt \(x(\J)\) \emph{Bahn}, \emph{Orbit}, \emph{Trajektorie} von \(x'(t)=f(x(t))\).

\subsection*{Erstes Integral}

Sei \(H \in C^1(D,\R)\).

\(H\) heißt \emph{erstes Integral} von \(x'(t)=f(x(t))\) gdw.: \[\forall x \in D : H'(x) \cdot f(x) = 0\]

Sei weiter \(x : \J \to \R^p\) Lsg. von \(x'(t)=f(x(t))\). Dann ex. \(c \in \R\) s.d.: \(\forall t \in \J : H(x(t))=c\)

\subsection*{Stabilität}

Sei \(x_0 \in D\) mit \(f(x_0)=0\) stat. Stelle von \(x'=f(x)\).

\spacing

\(x_0\) heißt \emph{stabil} gdw. \(\forall \epsilon > 0 \exists \delta > 0 : \|x_1-x_0\| < \delta\) und ist \(x : [t_0,\omega_+) \to \R^p\) die nach rechts nicht fort. Lsg. des AWP \(x'(t)=f(x(t)), x(t_0)=x_1\) so ist \(\omega_+ = \infty\) und \(\forall t \geq t_0 : \|x(t)-x_0\| < \epsilon\).

\spacing

Gilt \(x(t) \to x_0 \ (t \to \infty)\) so ist \(x_0\) asymp. stabil.

\subsubsection*{Stabilitätssatz}

Sei \(f(x_0)=0\) und \(f\) in \(x_0\) diffbar. Gilt für EW \(\forall \lambda \in \sigma(f'(x_0)) : \text{Re} \lambda < 0\) so ist \(x_0\) asymptotisch stabil.

\spacing

Gilt \(\exists \lambda \in \sigma(f'(x_0)) : \text{Re} \lambda > 0\) so ist \(x_0\) instabil.

\subsubsection*{Lyapunov-Funktion}

Sei \(\emptyset \neq D \subseteq \R^p\) offen und \(f : D \to \R^p\) lokal Lipschitz sowie \(x_0 = 0 \in D\) und \(f(x_0)=0\).

Für das AWP \(x'(t)=f(x(t)), \ x(t_0)=0\) ist def.:

\spacing

Sei \(r > 0, U_r(0) \subseteq D\) und \(V \in C^1(U_r(0),\R)\). \(V\) heißt \emph{Lyapunov-Funktion} zu \(x'=f(x)\) in Punkt \(x_0=0\) gdw.: \(V(0)=0, \forall x \in U_r(0) \setminus \{0\} : V(x) > 0\) und \(\forall x \in U_r(0) : V'(x) \cdot f(x) \leq 0\).

\spacing

Besitzt \(x'=f(x)\) eine LF so ist \(x_0=0\) stabil.

\spacing

Gilt weiter \(\forall x \in U_r(0) \setminus \{0\} : V'(x) \cdot f(x) < 0\) so ist \(x_0\) asymptotisch stabil.

\section*{Randwertprobleme}
