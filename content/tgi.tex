\renewcommand{\P}{\mathcal{P}}
\newcommand{\NP}{\mathcal{NP}}

\section*{Endliche Automaten}

Ein \emph{deterministischer endlicher Automat} besteht aus endlichen Mengen von Zuständen, Eingabesymbolen und einer Übergangsfunktion.

Er entscheidet ob eine endliche Eingabe gültig ist.

\subsection*{Reguläre Ausdrücke}

\emph{Reguläre Ausdrücke} beschreiben \emph{reguläre Sprachen}.

Dies sind genau die Sprachen, die nach dem \emph{Satz von Kleene} von einem DEA aktzeptiert werden.

\subsection*{Nichtdeterministische Automaten}

Zustandsübergänge sind nichtdeterministisch.

Jeder NEA besitzt einen äquivalenten DEA.

Gebildet mit \emph{Potenzmengenkostruktion}.

\subsubsection*{$\epsilon$-Übergänge}

Jeder NEA mit $\epsilon$-Übergängen besitzt einen äquivalenten NEA ohne $\epsilon$-Übergänge, der nicht mehr Zustände benötigt.

\subsection*{Pumping-Lemma für reguläre Sprachen}

Sei $L$ reguläre Sprache.

Dann $\exists n \in \N \forall w \in L : |w| > n \implies w=uvx$ mit $|uv| \leq n, v \neq \epsilon$ und $\forall i \in \N_0 : uv^ix \in L$.

\subsection*{Äquivalenzklassenautomat}

Nicht erreichbare Zustände in $Q$ sind \emph{überflüssig}.

Diese sind in $\mathcal{O}(|Q|\cdot|\Sigma|)$ bestimmbar.

Ein Automat ohne überflüssige Zustände ist nicht zwingend minimal.

$p, q \in Q$ sind \emph{äquivalent} ($p \equiv q$), wenn $\forall w \in \Sigma^* : \delta(p,w) \in F \iff \delta(q,w) \in F$.

\section*{Entscheidbarkeit}

Eine Sprache $L \subseteq \Sigma^*$ ist \emph{rekursiv / entscheidbar}, wenn es eine Turing-Maschine gibt, die auf allen Eingaben hält und $w \in \L$ aktzeptiert gdw. $w \in L$.

\spacing

Eine Sprache $L \subseteq \Sigma^*$ ist \emph{rekursiv-aufzählbar / semi-entscheidbar}, wenn es eine TM gibt, die $w \in L$ aktzeptiert. Ihr Verhalten für $w \neq L$ ist undefiniert.

\spacing

Eine Funktion $f : \Sigma^* \to \Gamma^*$ ist \emph{(Turing)-berechenbar / totalrekursiv}, wenn es eine TM gibt, die für $w \in \Sigma^*$ das Wort $f(w) \in \Gamma^*$ ausgibt.

\spacing

Eine Sprache $L \subseteq \Sigma^*$ ist \emph{entscheidbar} gdw. ihre \emph{charakteristische Funktion} berechenbar ist.

\subsection*{Church'sche These}

Die Menge der Turing-berechenbaren Funktionen ist genau die Menge der im intuitiven Sinne überhaupt berechenbaren Funktionen.

\subsection*{Universelle Turing-Maschinen}

Sei $\mathcal{M} := (Q,\Sigma,\Gamma,\delta,s,F)$ eine Turing-Maschine.

Ihre \emph{Gödelnummer} $\langle M \rangle$ ist definiert als:

\spacing

$\mathcal{M}$ wird durch $111\text{code}_1 11\text{code}_2 11 \dots 11\text{code}_z 111$ kodiert, $\text{code}_i$ stellt $z$ Funktionswerte von $\delta$ dar:

\spacing

Kodiere $\delta(q_i,a_j) = (q_r,a_s,d_t)$ mit $0^i10^j10^r10^s10^t$ wobei $d_t \in \{d_1,d_2,d_3\}$ für $L$, $R$ bzw. $N$ steht.

\spacing

\emph{Universelle Turing-Maschine} aktzeptiert $(\langle \mathcal{M} \rangle, w)$ und simuliert $\mathcal{M}$ auf $w$.

\subsubsection*{Diagonalsprache}

$T_w$ ist TM mit Gödelnummer $w \in \{0,1\}^*$.

Sei $w_i \in \{0,1\}^*$ für $i = 0,1\dots$.

\spacing

Die \emph{Diagonalsprache} ist definiert durch:

$L_d := \{ w_i | \mathcal{M}_i \text{ aktzeptiert } w_i \text{ nicht} \}$.

$L_d$ enthält Wörter $w_i$ die sich als Gödelnummer interpretiert nicht selbst aktzeptieren.

\spacing

$L_d$ und $L_d^c$ sind nicht entscheidbar.

\subsubsection*{Halteproblem}

\[ \mathcal{H} := \{ wv | T_w \text{ hält auf Eingabe } v \} \]

$\mathcal{H}$ ist nicht entscheidbar.

\subsubsection*{Universelle Sprache}

\[ L_u := \{ v \in L(T_w) \} \]

d.h. die Menge der Wörter $wv$ s.d. $T_w$ für Eingabe $v$ hält und diese aktzeptiert.

$L_u$ ist nicht entscheidbar aber semi-entscheidbar.

\subsubsection*{Satz von Rice}

Sei $R$ die Menge aller von TM berechenbaren Funktionen und $S \subseteq R$ nicht trivial. Dann:

\vspace*{-4mm}
\[ L(S) := \{ \langle\mathcal{M}\rangle | \mathcal{M} \text{ berechnet Funktion aus } S \} \]
$L(s)$ ist nicht entscheidbar.

\subsection*{(Semi-)entscheidbare Sprachen}

Entscheidbare Sprachen sind abgeschlossen unter Komplementbildung, Schnitt und Vereinigung.

\spacing

Semi-entscheidbare Sprachen sind abgeschlossen unter Schnitt und Vereinigung.

\section*{Komplexitätsklassen}

Sind nichtdeterministische TM wesentlich effizienter als deterministische TM? $\P \neq \NP$?

\subsection*{Nichtdeterministische Turing-Maschinen}

Übergangsfunktion $\delta$ bietet Wahlmöglichkeiten und $\epsilon$-Übergänge vergleichbar mit NEAs.

\subsection*{$\NP$-vollständige Probleme}

$\P \subseteq \NP$ trivial, $\P \neq \NP$ d.h. $\P \subset \NP$ vermutet.

\spacing

Eine \emph{polynomiale Transformation} von $L_1 \subseteq \Sigma_1^*$ nach $L_2 \subseteq \Sigma_2^*$ ist $f : \Sigma_1^* \to \Sigma_2^*$ s.d. eine DTM mit polynomialer Laufzeit existiert, die $f$ berechnet und $\forall x \in \Sigma_1^* : x \in L_1 \iff f(x) \in L_2$.

Geschrieben: $L_1 \propto L_2$.

\subsubsection*{$\NP$-Vollständigkeit}

Eine Sprache $L$ ist \emph{$\NP$-vollständig}, wenn $L \in \NP$ und $\forall L' \in  \NP : L' \propto L$.

\subsubsection*{Erfüllbarkeitsproblem (SAT)}

Prüfe ob Belegungen von booleschen Variablen existiert s.d. gegebene Klauseln erfüllt werden.

\spacing

\emph{SAT} ist $\NP$-vollständig. Insb. ist \emph{3SAT} für Klauseln mit genau drei Literalen $\NP$-vollständig.

\subsubsection*{Erfüllbarkeitsproblem (Max2SAT)}

Prüfe ob Belegung ex. s.d. mind. $K$ Klauseln mit jeweils genau zwei Literalen erfüllt werden.

\emph{Max2SAT} ist $\NP$-vollständig.

\subsubsection*{Cliquen in Graphen (CLIQUE)}

Prüfe ob Clique der Größe mind. $K$ existiert.

\emph{CLIQUE} ist $\NP$-vollständig.

\subsubsection*{Graphenfärbung (COLOR)}

Prüfe ob Knotenfärbung mit max. $K$ Farben ex.

\emph{3COLOR} ist $\NP$-vollständig.

\subsubsection*{EXACT COVER}

Sei $X$ endl. Menge und $\mathcal{S}$ Familie von Teilmengen.

Prüfe ob $\mathcal{S}' \subseteq \mathcal{S}$ ex. s.d. $\forall a \in X \exists! A \in \mathcal{S}' : a \in A$.

\emph{EXACT COVER} ist $\NP$-vollständig.

\subsubsection*{SUBSET SUM}

Sei $M$ endl. Menge, $w : M \to \N_0$ und $K \in \N_0$.

Prüfe ob $M' \subseteq M$ ex. s.d. $\sum_{a \in M'} w(a) = K$.

\emph{SUBSET SUM} ist $\NP$-vollständig.

\subsubsection*{PARTITION}

Sei $M$ endl. Menge und $w : M \to \N_0$.

Prüfe ob $M' \subseteq M$ ex. s.d. $\textstyle\sum\limits_{a \in M'} w(a) = \textstyle\sum\limits_{a \in M \setminus M'} w(a)$.

\emph{PARTITION} ist $\NP$-vollständig.

\subsubsection*{KNAPSACK}

Sei $M$ endl. Menge, $w : M \to \N_0$ Gewichtsfkt., $c : M \to \N_0$ Kostenfkt. und $W,C \in \N_0$.

Prüfe ob $M' \subseteq M$ existiert s.d. $\sum_{a \in M'} w(a) \leq W$ und $\sum_{a \in M'} c(a) \geq C$.

\emph{KNAPSACK} ist $\NP$-vollständig.

\subsection*{Suchprobleme}

\emph{Suchproblem} $\Pi$ ist geg. mit Menge von Beispielen $D_\Pi$ und für $I \in D_\Pi$ Menge $S_\Pi(I)$ aller Lsg. von $I$.

Die Lösung eines Suchproblems ist die Angabe von $S_\Pi(I)$ für $I \in D_\Pi$ mit $S_\Pi(I) \neq \emptyset$ falls möglich.

\spacing

Beispiele sind Bestimmung einer optimalen Tour in Graph (TSP) oder eines Hamilton-Kreises.

\subsection*{Aufzählungsprobleme}

\emph{Aufzählungsproblem} $\Pi$ ist geg. mit Menge von Beispielen $D_\Pi$ und für $I \in D_\Pi$ Menge $S_\Pi(I)$ aller Lsg.

Lösung eines Aufzählungsproblems ist $|S_\Pi(I)|$.

\spacing

z.B. wie viele Hamilton-Kreise gibt es?

\subsubsection*{$\NP$-Schwere}

Ein Suchproblem $\Pi$ ist \emph{$\NP$-schwer}, wenn $\NP$-vollständige Sprache $L$ ex. s.d. $L \propto_T \Pi$.

$\NP$-schweres Problem ist nicht zwingend in $\NP$.

Ein bel. Problem ist $\NP$-schwer, wenn es mind. so schwer ist, wie alle $\NP$-vollständigen Probleme.

\spacing

Die Klasse der $\NP$-schweren Probleme ist bzgl. Komplementbildung abgeschlossen.

\subsection*{INTEGER PROGRAMMING}

Sei $a_{ij}, b_i, c_j, b \in \Z_0$ für $1 \leq i \leq m$ und $1 \leq j \leq n$.

Prüfe ob $x_i, \dots, x_n \in \N_0$ ex. s.d. $\sum_{j=1}^n c_j \cdot x_j = B$ und $\forall 1 \leq i \leq m : \sum_{j=1}^n a_{ij} \cdot x_j \leq b_j$ (d.h. $A \cdot \vec{x} \leq \vec{b}$).

\emph{INTEGER PROGRAMMING} ist $\NP$-schwer.

\section*{Chomsky-Hierarchie}

\begin{description}[leftmargin=!,labelwidth=8mm]
	\item[Typ 0] Rekursiv aufzählbar, Turing-Maschine
	\item[Typ 1] Kontextsensitiv, nichtdet. TM
	\item[Typ 2] Kontextfrei, nichtdet. Kellerautomat
	\item[Typ 3] Regulär, Endlicher Automat (DEA, NEA)
\end{description}

\subsection*{Chomsky-Normalform}
