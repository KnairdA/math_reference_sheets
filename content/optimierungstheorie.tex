\section*{Optimierungsprobleme}

Sei \(f: M \to \R\). Ein Optimierungsproblem ist:

\((P) \ \min_{x \in M} f(x)\)

\((P)\) ist zulässig, wenn \(M \neq \emptyset\).

\(\exists \hat x \in M \forall x \in M : f(\hat x) \leq f(x)\), so ist \((P)\) lösbar.

\subsection*{Klassifizierung}

\begin{enumerate}[label=(\alph*)]
	\item Endlich dim. Probleme mit \(M \subseteq \R^m\)
	\item Lineare Probleme (vgl. LP)
	\item Konvexe Probleme
	\item Differenzierbare Probleme
\end{enumerate}

\subsection*{Lineare Programme}

\((P) \ \min_{x \in M} f(x)\) ist linear, wenn:

\(f: \R^n \to \R\) affin linear ist mit \(f(x)=c^T x + c_0\) für \(c \in \R^n, c_0 \in \R\) und \(M\) folgende Darstellung besitzt:

\(M = \{ x \in \R^n | A_g x = b_g, A_u x \leq b_u \}\) mit \\ \(A = (A_g \ A_n)^T \in \R^{(m+p) \times n}, b = (b_g \ b_u)^T \in \R^{m+p}\)

\subsubsection*{Normalform}

Ein LP \((P_N)\) ist in Normalform gegeben, wenn \(A \in \R^{m \times n}, b \in \R^m\) und \(c \in \R^n\) ex. s.d. gilt:

\((P_N) \ \min c^T x\) mit \(M_N = \{ x \in \R_{\geq 0}^n | Ax=b \}\)

\spacing

Jedes LP \((P)\) besitzt Normalform \((P_N)\).

Dafür werden Ungleichungsbedingungen über \emph{Schlupfvariablen} umgeformt:

\(A_u x \leq b_u \rightsquigarrow x_i^s = (b_u)_i - (A_u x)_i\) für \(i = 1,\dots,s\).

\subsubsection*{Konvexe Mengen}

\(U \subseteq \R^n\) ist \emph{konvex}, wenn für \(x, y \in U \land \lambda \in [0,1] : (1-\lambda)x + \lambda y \in U\) gilt.

\spacing

\( v = \sum_{j=1}^m \lambda_j v^{(j)} \) mit \( \lambda_j \in [0,1] \land \sum_{j=1}^m \lambda_j = 1\) heißt \emph{Konvexkombination} von \( v^{(1)}, \dots, v^{(m)} \).

\spacing

\((M_N)\) der Normalform ist konvex.

\spacing

Schnitte endlich vieler Halbräume in \(R^n\) heißen \emph{Polyeder}, beschränkte Polyeder heißen \emph{Polytope}.

\spacing

\(M_n\) ist Polytop gdw. \( \not\exists y \in \R_{\geq 0}^n \setminus \{0\} : Ay = 0 \).

\subsubsection*{Charakterisierung von Ecken in \(M_N\)}

\(x \in M\) ist Ecke, wenn aus \(x=(1-\lambda) u+\lambda v\) mit \(u, v \in M, \lambda \in (0,1)\) folgt: \( u = v = x \).

Ecke ist nicht als echte Konvexkomb. darstellbar.

\spacing

\( x \in M_N \) ist Ecke von \(M_N\) gdw. die Spalten \(\{a_{*j}\}_{j \in J_x}\) mit \(J_x = \{ j \in \{1,\dots,n\} | x_j > 0 \}\) linear unabhg. sind.

\subsubsection*{Satz zur Existenz endl. vieler Ecken in \(M_N\)}

\[ M_N \neq \emptyset \implies \exists x \in M_N : x \text{ ist Ecke} \]

Insgesamt existieren endlich viele Ecken.

\subsubsection*{Konvexkombination von Ecken}

Seien \( v^{(k)} \in M_N \) mit \(k=1,\dots,K\) Ecken von \(M_N\).

Dann \( \forall x \in M_N \exists \alpha_j \in [0,1], y \in \{ y \in \R_{\geq 0}^n | Ay = 0 \} :\)

\( \sum_{j=1}^K \alpha_j = 1 \land x = \sum_{j=1}^K \alpha_j v^{(j)} + y \)
