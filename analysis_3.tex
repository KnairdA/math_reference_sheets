\section*{$\sigma$-Algebren}

Ein Mengensystem $\mathcal{A} \subseteq \mathcal{P}(X)$ ist $\sigma$-Algebra auf der nichtleeren Menge $X$ gdw.:

\begin{enumerate}[label=(\alph*)]
	\item $X \in \mathcal{A}$
	\item $A \in \mathcal{A} \Rightarrow A^c := X\setminus A \in \mathcal{A}$
	\item $\forall j \in \mathbb{N} : A_j \in \mathcal{A} \Rightarrow \bigcup_{j\in \mathbb{N}} A_j \in \mathcal{A}$
\end{enumerate}

\subsection*{Eigenschaften von $\sigma$-Algebren}

Seien $\mathcal{A}$ eine $\sigma$-Algebra auf $X$, $n \in \mathbb{N}$, $\forall j \in \mathbb{N} : A_j \in \mathcal{A}$, dann ist $\mathcal{A}$ nach den folgenden Eigenschaften abgeschlossen unter abzählbaren Mengenoperationen:

\begin{enumerate}[label=(\alph*)]
	\item $\emptyset = X^c \in \mathcal{A}$
	\item $A_1 \bigcup \cdots \bigcup A_n \in \mathcal{A}$
	\item $A_1 \bigcap \cdots \bigcap A_n \in \mathcal{A}$
	\item $\bigcap_{j\in \mathbb{N}} A_j \in \mathcal{A}$
	\item $A_1 \setminus A_2 := A_1 \bigcap A_2^c \in \mathcal{A}$
\end{enumerate}

\subsection*{Erzeugte $\sigma$-Algebren}

Die durch das nichtleere Mengensystem $\mathcal{E} \subseteq \mathcal{P}(X)$ auf $X$ erzeugte $\sigma$-Algebra ist wie folgt definiert:

\vspace*{-4mm}
$$\sigma(\mathcal{E}) := \bigcap\{ \mathcal{A} \subseteq \mathcal{P}(X) | \mathcal{A} \text{ ist } \sigma \text{-Algebra}, \mathcal{E} \subseteq \mathcal{A} \}$$

Der Erzeuger $\mathcal{E}$ ist hierbei allg. nicht eindeutig.

\subsubsection*{Eigenschaften erzeugter $\sigma$-Algebren}

Sei $\emptyset \neq \mathcal{E} \subseteq \mathcal{P}(X)$, dann gilt:

\begin{enumerate}[label=(\alph*)]
	\item $\mathcal{A}$ ist $\sigma$-Algebra $\land$ $\mathcal{E} \subseteq \mathcal{A} \Rightarrow \mathcal{E} \subseteq \sigma(\mathcal{E}) \subseteq \mathcal{A}$
	\item $\sigma(\mathcal{E})$ ist kleinste $\mathcal{E}$ enthaltende $\sigma$-Algebra.
	\item $\mathcal{E}$ ist $\sigma$-Algebra $\Rightarrow \mathcal{E} = \sigma(\mathcal{E})$
	\item $\mathcal{E} \subseteq \mathcal{E}' \subseteq \mathcal{P}(X) \Rightarrow \sigma(\mathcal{E}) \subseteq \sigma(\mathcal{E}')$
\end{enumerate}

\subsection*{Borelsche $\sigma$-Algebra}

Sei $X$ ein metrischer Raum und $\mathcal{O}(X)$ das System der in $X$ offenen Mengen, dann ist $\mathcal{B}(X) := \sigma(\mathcal{O}(X))$ die Borelsche $\sigma$-Algebra auf $X$.

Im Speziellen wird $\mathcal{B}_m := \mathcal{B}(\mathbb{R}^m)$ gesetzt.

$\mathcal{B}_m$ enthält insb. alle offenen und abgeschlossenen Mengen in $\mathbb{R}^m$ sowie deren abzählbaren Vereinigungen und Durchschnitte.

\subsubsection*{Charakterisierung}

\vspace*{-4mm}
\begin{align*}
	\mathcal{B}_m &= \sigma(\{(a, b) | a, b \in \mathbb{Q}^m, a \leq b\}) \\
	              &= \sigma(\{(a, b] | a, b \in \mathbb{Q}^m, a \leq b\})
\end{align*}
