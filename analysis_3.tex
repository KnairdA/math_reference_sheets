\section*{Nützliches aus der Mengenlehre}

\subsection*{De Morgansche Regeln}

Sei $\mathcal{B}$ ein Mengensystem.

$$\left(\bigcup_{B\in \mathcal{B}} B \right)^c = \bigcap_{B\in \mathcal{B}} B^c \hspace*{8mm} \left(\bigcap_{B\in \mathcal{B}} \right)^c = \bigcup_{B\in \mathcal{B}} B^c$$

\subsection*{Mengen-Ring}

Ein Mengensystem $\mathcal{A}$ ist ein Ring gdw. $\forall A, B \in \mathcal{A}$:

\begin{enumerate}[label=(\alph*)]
	\item $\emptyset \in \mathcal{A}$
	\item $B\setminus A \in \mathcal{A}$
	\item $A \cup B \in \mathcal{A}$
\end{enumerate}

\section*{$\sigma$-Algebren}

Ein Mengensystem $\mathcal{A} \subseteq \mathcal{P}(X)$ ist $\sigma$-Algebra auf der nichtleeren Menge $X$ gdw.:

\begin{enumerate}[label=(\alph*)]
	\item $X \in \mathcal{A}$
	\item $A \in \mathcal{A} \Rightarrow A^c := X\setminus A \in \mathcal{A}$
	\item $\forall j \in \N : A_j \in \mathcal{A} \Rightarrow \bigcup_{j\in \N} A_j \in \mathcal{A}$
\end{enumerate}

\subsection*{Eigenschaften von $\sigma$-Algebren}

Seien $\mathcal{A}$ eine $\sigma$-Algebra auf $X$, $n \in \N$, $\forall j \in \N : A_j \in \mathcal{A}$, dann ist $\mathcal{A}$ nach den folgenden Eigenschaften abgeschlossen unter abzählbaren Mengenoperationen:

\begin{enumerate}[label=(\alph*)]
	\item $\emptyset = X^c \in \mathcal{A}$
	\item $A_1 \bigcup \cdots \bigcup A_n \in \mathcal{A}$
	\item $A_1 \bigcap \cdots \bigcap A_n \in \mathcal{A}$
	\item $\bigcap_{j\in \N} A_j \in \mathcal{A}$
	\item $A_1 \setminus A_2 := A_1 \bigcap A_2^c \in \mathcal{A}$
\end{enumerate}

\subsection*{Erzeugte $\sigma$-Algebren}

Die durch das nichtleere Mengensystem $\mathcal{E} \subseteq \mathcal{P}(X)$ auf $X$ erzeugte $\sigma$-Algebra ist wie folgt definiert:

\vspace*{-4mm}
$$\sigma(\mathcal{E}) := \bigcap\{ \mathcal{A} \subseteq \mathcal{P}(X) | \mathcal{A} \text{ ist } \sigma \text{-Algebra}, \mathcal{E} \subseteq \mathcal{A} \}$$

Der Erzeuger $\mathcal{E}$ ist hierbei allg. nicht eindeutig.

\subsubsection*{Eigenschaften erzeugter $\sigma$-Algebren}

Sei $\emptyset \neq \mathcal{E} \subseteq \mathcal{P}(X)$, dann gilt:

\begin{enumerate}[label=(\alph*)]
	\item $\mathcal{A}$ ist $\sigma$-Algebra $\land$ $\mathcal{E} \subseteq \mathcal{A} \Rightarrow \mathcal{E} \subseteq \sigma(\mathcal{E}) \subseteq \mathcal{A}$
	\item $\sigma(\mathcal{E})$ ist kleinste $\mathcal{E}$ enthaltende $\sigma$-Algebra.
	\item $\mathcal{E}$ ist $\sigma$-Algebra $\Rightarrow \mathcal{E} = \sigma(\mathcal{E})$
	\item $\mathcal{E} \subseteq \mathcal{E}' \subseteq \mathcal{P}(X) \Rightarrow \sigma(\mathcal{E}) \subseteq \sigma(\mathcal{E}')$
\end{enumerate}

\subsection*{Borelsche $\sigma$-Algebra}

Sei $X$ ein metrischer Raum und $\mathcal{O}(X)$ das System der in $X$ offenen Mengen, dann ist $\mathcal{B}(X) := \sigma(\mathcal{O}(X))$ die Borelsche $\sigma$-Algebra auf $X$.

Im Speziellen wird $\mathcal{B}_m := \mathcal{B}(\R^m)$ gesetzt.

$\mathcal{B}_m$ enthält insb. alle offenen und abgeschlossenen Mengen in $\R^m$ sowie deren abzählbaren Vereinigungen und Durchschnitte.

\subsubsection*{Charakterisierung}

\vspace*{-4mm}
\begin{align*}
	\mathcal{B}_m &= \sigma(\{(a, b) | a, b \in \mathbb{Q}^m, a \leq b\}) \\
	              &= \sigma(\{(a, b] | a, b \in \mathbb{Q}^m, a \leq b\})
\end{align*}

\section*{Maße auf $\sigma$-Algebren}

Sei $\mathcal{A}$ eine $\sigma$-Algebra auf $X$.

$\mu : \mathcal{A} \rightarrow [0, \infty]$ ist positives Maß auf $\mathcal{A}$ gdw.:

\begin{enumerate}[label=(\alph*)]
	\item $\mu(\emptyset) = 0$
	\item $\forall \text{ disjunkte } \{A_j | j \in \N\} \subseteq \mathcal{A} :\\ \hspace*{4mm} \mu(\dot\bigcup_{j\in \N} A_j) = \sum_{j\in \N} \mu(A_j)$
\end{enumerate}

\subsection*{Maßraum}

Ein Tripel $(X, \mathcal{A}, \mu)$ ist Maßraum. Ein endlicher Maßraum erfüllt zusätzlich $\mu(X) < \infty$.

Ein Wahrscheinlichkeitsmaß erfüllt $\mu(X) = 1$.

\subsection*{Punkt- / Diracmaß}

Für fest gewählte $\mathcal{A} = \mathcal{P}(X)$, $x \in X$ ist ein Wahrscheinlichkeitsmaß für $A \subseteq X$ definiert:

$$\delta_x(A) := \begin{cases}
	1 & x \in A \\
	0 & x \notin A
\end{cases}$$

Dieses wird Punkt- / Diracmaß auf $\mathcal{A}$ genannt.

\subsection*{Zählmaß}

Sei $\mathcal{A} = \mathcal{P}(\N)$ und $\forall j \in \N : p_j \in [0, \infty]$ fest gewählt.

$\mu(A) := \sum_{j\in A} p_j$ für $A \subseteq \N$ ist Maß auf $\mathcal{P}(\N)$.

Gilt zusätzlich $\forall j \in \N : p_j = 1$ so heißt $\mu$ Zählmaß.

\subsection*{Eigenschaften von Maßen}

Sei $(X, \mathcal{A}, \mu)$ Maßraum und $A, B, A_j \in \mathcal{A}$ für $j \in \N$.

\begin{description}[leftmargin=!,labelwidth=26mm]
	\item[Monotonie] $A \subseteq B \Rightarrow \mu(A) \leq \mu(B)$
	\item[$\sigma$-Subadditivität] $\mu(\dot\bigcup_{j\in \N} A_j) \leq \sum_{j\in \N} \mu(A_j)$
	\item[Stetigkeit (unten)] $A_j \uparrow \Rightarrow \displaystyle\lim_{j\to \infty} \mu(A_j) = \mu(\bigcup_{j\in \N} A_j)$
	\item[Stetigkeit (oben)] $A_j \downarrow \land \hspace*{1mm} \mu(A_1) < \infty \\ \hspace*{4mm}\Rightarrow \displaystyle\lim_{j\to \infty} \mu(A_j) = \mu(\bigcap_{j\in \N} A_j)$
\end{description}

Für $\mu(A) < \infty$ folgt $\mu(B\setminus A) = \mu(B) - \mu(A)$.

Für endliche Maße gilt insb. $\mu(A^c) = \mu(X) - \mu(A)$.

\subsection*{Prämaß}

Eine Abb. $f : \mathcal{A} \rightarrow [0, \infty)$ ist ein Prämaß auf Ring $\mathcal{A}$ gdw.:

\begin{enumerate}[label=(\alph*)]
	\item $\mu(\emptyset) = 0$
	\item $\{A_j | j \in \N\} \subseteq \mathcal{A}$ disjunkt und $A = \bigcup_{j\in \N} A_j \in \mathcal{A} \Rightarrow \mu(A) = \sum_{j\in \N} \mu(A_j)$
\end{enumerate}

\section*{Lebesguemaß}

\subsection*{System der Intervalle}

Sei $I = (a, b] \subseteq \R^m$ für $a, b \in \R^m$ mit $a \leq b$, dann wird das System von Intervallen $\mathcal{J}_m$ definiert:

$\lambda(I) = \lambda_m(I) := (b_1 - a_1) \cdot \hdots \cdot (b_m - a_m)$

\subsection*{Ring der Figuren}

$$\mathcal{F}_m = \left\{ A = \bigcup_{j=1}^n I_j | I_j \in \mathcal{J}_m, n \in \N \right\}$$

\subsubsection*{Eigenschaften}

Seien $I_1, I_2 \in \mathcal{J}_m$:

\begin{enumerate}[label=(\alph*)]
	\item $\sigma(\mathcal{F}_m) = \mathcal{B}_m$
	\item $I_1 \cap I_2 \in \mathcal{J}_m$
	\item $I_1 \setminus I_2 \in \mathcal{F}_m$ sowie endliche Vereinigung disjunkter Intervalle aus $\mathcal{J}_m$
	\item $\forall A \in \mathcal{F}_m: A$ ist endliche Vereinigung disjunkter Intervalle aus $\mathcal{J}_m$
	\item $\mathcal{F}_m$ ist Ring
\end{enumerate}

\section*{Messbare Funktionen}

Sei $\mathcal{A}$ eine $\sigma$-Algebra auf $X\neq \emptyset$ und $\mathcal{B}$ eine $\sigma$-Algebra auf $Y\neq \emptyset$ sowie $f : X \rightarrow Y$ Funktion.

$f$ heißt ($\mathcal{A}$-$\mathcal{B}$-)messbar gdw. $\forall B \in \mathcal{B} : f^{-1}(B) \in \mathcal{A}$

\subsection*{Borel-Messbarkeit}

Seien $X, Y$ metrische Räume.

Die Funktion $f : X \rightarrow Y$ heißt Borel-messbar, wenn sie $\mathcal{B}(X)$-$\mathcal{B}(Y)$-messbar ist.

\subsection*{Eigenschaften}

Seien $\mathcal{A}, \mathcal{B}, \mathcal{C}$ $\sigma$-Algebren auf $X, Y, Z \neq \emptyset$.

\begin{enumerate}[label=(\alph*)]
	\item $f : X \rightarrow Y$ ist $\mathcal{A}$-$\mathcal{B}$-mb., $g : Y \rightarrow Z$ ist $\mathcal{B}$-$\mathcal{C}$-mb. $\Rightarrow g \circ f : X \rightarrow Z$ ist $\mathcal{A}$-$\mathcal{C}$-mb.
	\item $\emptyset \neq \mathcal{E} \subseteq \mathcal{P}(Y)$, $\mathcal{B} = \sigma(\mathcal{E})$, $f: X \rightarrow Y$ dann ist $f$ messbar gdw. $\forall E \in \mathcal{E} : f^{-1}(E) \in \mathcal{A}$
	\item $X, Y$ metrische Räume, $f : X \rightarrow Y$ stetig $\Rightarrow f$ ist Borel-messbar
	\item $f : X \rightarrow \R^m$ ist $\mathcal{A}$-$\mathcal{B}_m$-mb. gdw. $\forall i \in \{1, \dots, m\} : f_i : X \rightarrow \R$ ist $\mathcal{A}$-$\mathcal{B}_1$-mb.
	\item $f, g$ sind $\mathcal{A}$-$\mathcal{B}_1$-mb. und $\alpha, \beta \in \R \Rightarrow fg : X \rightarrow \R$ und $\frac{1}{f} : \{x \in X | f(x) \neq 0\} \rightarrow \R$ mb.
	\item $f : X \rightarrow \R^m$ ist $\mathcal{A}$-$\mathcal{B}_m$-mb. $\\\Rightarrow g : X \rightarrow \R; x \mapsto |f(x)|_2$ ist $\mathcal{A}$-$\mathcal{B}_1$-mb.
	\item $X = W \dot\cup Z$ mit $\emptyset \neq W, Z \in \mathcal{A}$, $f : W \rightarrow Y$ ist $\mathcal{A}_W$-$\mathcal{B}$-mb., $g : Z \rightarrow Y$ ist $\mathcal{A}_Z$-$\mathcal{B}$-mb. $\Rightarrow h(x) = \begin{cases}
	f(x) & x \in W \\
	g(x) & x \in Z
\end{cases}$ ist $\mathcal{A}$-$\mathcal{B}$-mb.
\end{enumerate}
